\documentclass{mcmthesis}
\mcmsetup{CTeX = true,  
        tcn = 1912947, problem = B,                                           
        sheet = true, titleinsheet = true, keywordsinsheet = true,
        titlepage = false, abstract = false}
\usepackage{multirow}
\usepackage{palatino} 
\usepackage{lipsum}
\usepackage{indentfirst}
\usepackage{amsmath}
\usepackage{amssymb}
\usepackage{graphicx, subfig}
\usepackage{caption}
\usepackage{booktabs}
\usepackage[section]{placeins}
\usepackage{geometry}
\usepackage{float}
\usepackage{algorithm}
\usepackage{algorithmic}
\usepackage{setspace}

%\geometry{bottom=2.5cm}
\geometry{left=2cm,right=2cm,top=2.5cm,bottom=2cm}
%\newcommand{\upcite}[1]{\textsuperscript{\textsuperscript{\te{#1}}}}   

%\setlength\parskip{.5\baselineskip}
%\newtheorem{definition}{Definition}[section]

\title{Creative Design Based on Transportable Disaster Response System(DroneGo)}                                        
%\author{\small \href{http://www.latexstudio.net/}
%  {\includegraphics[width=7cm]{mcmthesis-logo}}}
%\date{\today}
\begin{document}
\begin{abstract}  
	How to transport medical packages and finish reconnaissance task quickly when facing worst disaster and isolated environments is a valuable issue. In this paper, we propose a transportable disaster response system named DroneGo to model this real problem based on the condition of Puerto Rico's hurricane in 2017. \par
	Grids Traversing Model (GTM) is proposed as a basic quantification of the real problem, since the calculation for delivery and detection tasks can be conveniently done within a grids environment. For determining which grid to be detected, we take population density, traffic high way density into calculation for grids weights matrix with the help of Entropy Weighting Model (EWM). The generated matrix and maps benefit the work of flight plan and location. \par
	By using EWM again, we rank the seven types of drones to find out the best drone for delivery and for reconnaissance, which greatly simplify the scope of problems. Then we adopt a heuristic algorithm to solve the 3D container-packing problem and obtain the packing solution with maximum spatial utilization rate.\par
	For the optimum localization problem, we starts from five hospitals and consider all the drone types, screening out B, C, F Drone. Then we analysis three kinds of flight path and obtain the combination of circle map and ellipse map. By constructing two kinds of coverage rates (area and weight) and aiming for increasing them, we figure out three optimum positions for three cargo containers.    \par
	Finally comes the most complex flight plan, which is based on all the previous conclusion. We try our best to simulate three flight plan of containers in A* Algorithm and get some considerable achievements - we find our solution can reconnoitre 83.61\% area of Puerto Rico under the  precondition that three day's delivery task can be guaranteed. 
	
	\begin{keywords}
	Grids Traversing Model; 3D-container-packing Problem; Entropy  \\ \enspace\enspace\indent\enspace\enspace\indent\enspace\enspace\indent\enspace\enspace\enspace   Weighting Model;  Optimum Location Problem; A* Algorithm  
	
	\end{keywords}
\end{abstract}
%\enspace\enspace\indent\enspace\enspace 
    \maketitle                             

    \newpage                              
    \setcounter{page}{1}                    
    \tableofcontents                       
    \newpage                             

    %\section{Background Introduction}
In the age of Internet dating, there are more romantic options than fish in the well. Many daters believe that having more options means they're more likely to find the right person for they while many daters find that less romantic options may lead to better outcomes without much anxiety. In another view, when faced with a myriad of choices, the pleasure at the prospect of more options is canceled out by the anticipated loss of making a wrong choice.  Actually, research has found that speed daters often choose their partners based on their looks. But when faced with fewer options, daters are likely to take the time to reflect on a person's deeper qualities. That suggests that in order to evaluate the qualities that matter -- which, for most people, are things like a partner’s honesty, his dependability, her sense of humor—going deeper in search but not wider is necessary.\par
Considering these phenomena, some feasible and scientific algorithms or methods can be applied to recommend appropriate partners for people want to find boy friend or girl friend.

\section{The Description of Problem}   
\subsection{Problem One: Online Dating Matches  }  
The request of problem one is concise: Create an objective quantitative algorithm or set of algorithms to complete online dating matches by few options. There are two steps we can follow:

\begin{enumerate}
	\item \textbf{Unsupervised Learning}: Since the data set of human beings can be very huge, we can filter all the human sample data to several classes, which is irrelevant to the attributes of data. This process can be regarded as unsupervised learning, so K-means algorithm can be considered. 
	\item \textbf{Supervised Learning}: According to data on few options, some objective functions and certain constraints could be set up. Aimming at the online dating matches, some algorithms in personalization recommendation such as K-Nearest Neighbor, Matrix Decomposition Recommender System, User-Based Collaborative Filtering, Model-Based Collaborative Filtering and Psychology-Based Recommendation can all be used in our dating matches models. This process can be regarded as supervised learning.
\end{enumerate}

\subsection{Problem Two:  More Suitable Estimate of An Ideally Sized Choice Set}                 
The goal of this problem is to give a more suitable estimate of an ideally sized choice set in the development of "Top 20 Recommended Daters" list established from problem one. Some Manual analysis can be done on the information of the top 20 daters from more angles and finally a best daters visualized report of eight persons can be generated and presented to the user to see his or her ideal daters with variety and depth consideration. 


\subsection{Problem Three: Information Forms Design and Effect Analysis Study}
This problem consists of two parts:
\begin{quote}
\begin{itemize}
	\item Design the information forms.
	\item Study the relationship between forms design and success rate of online dating. 
\end{itemize}
\end{quote}
\par
For the design of forms, apart from some necessary information that every user must give, different kinds of questions can be designed to different users according to their unique personality traits. Besides, different size, style of forms can all be adopted to gather detailed information from users indirectly.\par
For the relationship study, a relational matrix between quantitative analysis of forms and the success rate of online dating can be established and then this problem is transferred to an regression problem which can be solved in quite a lot ways such as Linear Regression, Logistic Regression, Polynomial Regression, Stepwise Regression or even Support Vector Machine.

\subsection{Problem Four:  Non-Technical News Release }
 Problem four requires to write a one-page non-technical News Release describing the algorithms used, the results, and the website designed, which is based on the first three problems. With the limitation of one page, some refinement and streamlined form of expressions must be considered.  

\section{Terminology Explained in Model \cite{1}}      
\begin{itemize}        
	\item Unsupervised Learning: A branch of machine learning that learns from test data that has not been labeled, classified or categorized. Instead of responding to feedback, unsupervised learning identifies commonalities in the data and reacts based on the presence or absence of such commonalities in each new piece of data. 
	\item Supervised learning: The machine learning task of learning a function that maps an input to an output based on example input-output pairs. A supervised learning algorithm analyzes the training data and produces an inferred function, which can be used for mapping new examples.
	\item Loss Function and Objective Function: In mathematical optimization, statistics, econometrics, decision theory, machine learning and computational neuroscience, a loss function or cost function is a function that maps an event or values of one or more variables onto a real number intuitively representing some "cost" associated with the event. An optimization problem seeks to minimize a loss function. An objective function is either a loss function or its negative (in specific domains, variously called a reward function, a profit function, a utility function, a fitness function, etc.), in which case it is to be maximized. 

\end{itemize}


\section{Assumptions}
1. Assume that all the data needed in models, algorithms or forms design are available in reality.

2. All data we want to collect are selected based on whether is useful for correct daters recommendation without privacy considerations. 

3. Assume that data from users are all true and reliable and no need of noise elimination.

4. No small probability or random events in the recommendation system, such as world war and economic crisis.

5. Assume that personality traits of users do not change significantly over time.

\section{Symbols and Definitions}

\begin{center}
	\begin{tabular}{|p{80pt}|c|p{80pt}|}
		\hline
		\makebox[0.15\textwidth][c]{\textbf{symbol}}	& \makebox[0.1\textwidth][c]{\textbf{Meanings}} \\ \hline
	    $\mathbf{Scores_{i,j}}$        &  score of j\_th person through i\_th person's eyes         \\ \hline
		m       & number of male users           \\ \hline
		w       & number of female users  \\ \hline
		u       & number of users, $u=m+w$    \\ \hline
	    N  & the top N recommended daters    \\ \hline
	    onum     &   number of options selected in a specific situation \\ \hline
	   \textbf{ User\_Item} &  a matrix with shape(u,), column number is optional. \\ &
	    $User_i$\_$Item_j$ means the value of i\_th user on j\_th item \\ \hline
	    \textbf{Top}    &    a matrix storing information of recommended daters.\\ \hline
	\end{tabular}
\end{center}
%\begin{figure}[!htbp]                                       
%	\centering
%	\includegraphics[width = .8\textwidth]{ourwork.jpg}      
%	\caption{Flow Chart}                          
%	\label{ourwork}                                      
%\end{figure}
%\begin{figure}[!htbp]                                      
%	\centering
%	\includegraphics[width = .8\textwidth]{introduction.jpg}      
%	\caption{The Distribution of Various Language}                                  
%	\label{introduction}                                           
%\end{figure}
\section{Problem Solutions with Model Foundation}  
\subsection{Solution for Problem One} 
\subsubsection{General Idea}                              

Since the main task of this problem is to complete online dating matches, we can regard it as a recommendation system problem. We name the system \textbf{Perfect Dater Partner Online Recommendation System}. The core of this system is the dater match algorithm which can analyse the information of users inputting online and quickly recommend an ideally sized object choice set to users. \par
Nowadays, there are many mature recommendation algorithms in online shopping, online movies which can recommend perfectly appropriate goods. But when it comes to recommend daters, things become serious and more complex. 
\indent
\paragraph{Object Matrix}
For evaluating which person is a ideal object for a user called $U1$, we can construct a two-dimensional square matrix named \textbf{Scores}. The shape of \textbf{Scores} is $(u,u)$, we have to obtain the final Scores between every pair of users including each pair of two men and two women. This is a general solution which can calculate all the probability of two person regardless of their sex. In reality, we may divide the user data set to several parts according to their sexual orientation, in which the shape could be $(m,w), (w,m), (m,m), (w,w), (w,u), (m,u)$. But for universality and simplification, we only consider the \textbf{Scores} matrix with shape $(u,u)$ as our evaluation matrix. The element $\mathbf{Scores_{i,j}}$ means the score of j\_th person through i\_th person's eyes. Through comparing and quickly sorting(Quick Sort Algorithm) the Scores in one row, the top \textbf{N} daters of $U1$ are easily founded. 

\paragraph{Few Options}
According to the meaning of problem and for simplification, use few options as the inputs of algorithms. Due to different options needed by different algorithms, the detailed options will be shown in individual model.

\paragraph{Similarity Measurement \label{sm}}
A significant aspect is the way to measure the gap of two people which could be references of constructing objective function. The distance(or similarity) measurement methods we can use are very abundant\cite{2}:\\
\begin{tabular}{p{0.3\linewidth}p{0.4\linewidth}p{0.3\linewidth}}
	\textbullet \ \textbf{ Euclidean Distance} & \textbullet \  Manhattan Distance  & \textbullet \ Chebyshev Distance \\
	\textbullet \ Minkowski Distance & \textbullet \   Mahalanobis Distance & \textbullet \  \textbf{Cosine Distance}\\
	\textbullet \ Information Entropy & \textbullet \ Hamming Distance & \textbullet \ Jaccard Distance \\
	\textbullet \ 	Correlation Distance & \textbullet \ Standardized Euclidean Distance
\end{tabular}
For example, Euclidean Distance can be presented as followed:
\begin{equation}
{\textrm{ED}}(u_1,u_2) = \sqrt {{{(u{_{11}} - u_{21})}^2} + {{(u_{12} - u_{22})}^2} + ... + {{(u{_{1onum}} - u{_{2onum}})}^2}}
\end{equation}
$u_1, u_2$ mean two users to be evaluated; $u_{1i}$ means the score of $u_1$ on the i\_th option; $onum$ means the number of options selected.\\
Another example, Cosine Distance:
\begin{equation}
\cos ({u_1},{u_2}) = \frac{{\sum\limits_{i = 1}^{onum} {{u_1}_i{u_2}_i} }}{{\sqrt {\sum\limits_{i = 1}^{onum} {{u_1}_i^2} } \sqrt {\sum\limits_{i = 1}^{onum} {{u_2}{{_i}^2}} } }}
\end{equation}
All the distance measurement methods can be used to calculate the similarity of two persons. Training and testing work are needed as for which method is the best idea. \par
Our groups have tried to use some suitable algorithms based on daters recommendation, just see them in next several sections.

\subsubsection{Model 1: K-means}

This algorithm is only used for processing big data set. It can divide the whole data set into K clusters according to their global attributes distribution. The similar samples will be classified into the same cluster so that daters can be easier to find in the cluster they belong. 

The core of this algorithm is the way to measure the gap and initialize the positions of K centroids. For gap measurement, any similarity measurement method discussed in \ref{sm} are accepted. For the initialization of centroids, the following formula can be adopted:
\begin{equation}
	 centroidSet[i,j] = min(j)+(max(j)-min(j))*random(0,1)
\end{equation}
$centroidSet[i,j]$ means the value of j\_th option in i\_th centroid vector; $min(j)$ and $max(j)$ means the minimum and maximum value under the j\_th option; $random(0,1)$ means a random number between 0 and 1.

\paragraph{Inputs} A numerical matrix: \textbf{User\_Item}. The number of rows is u. The number of columns could be very large. Each value shows a user's evaluation or attribute about a item. A item could be any goods, things and any personal information.  
\paragraph{Outputs} A two-dimensional matrix named \textbf{User\_Class} with shape (u,2). The first column represents the ID of user and the second presents the class number to which this user belongs. Each user has a unique class number.

\paragraph{Application}
Since we do not have any big data set, an example program of K-means applied in \textbf{UCI Iris} data set is shown in appendix \ref{apA}. It is based on python language version 3.6.6 and run on jupyter notebook(a kind of Integrated Development Environment). We use the Euclidean Distance as the measurement of the gap between two samples and assign the K parameter to 3. Finally, the data set is divided to 3 clusters.

\paragraph{Strength and Weakness}
\subparagraph{Strength}
\begin{itemize}
	\item It's a unsupervised learning process without consideration of options.
	\item Can easily and scientifically divide the data set into K sets, which is helpful for processing big data sets.
	\item No need of complex operations. The calculation is simple and clear.
	\item Time complexity(O(uKt), t is the iteration number) and space complexity(O(u*onum)) is nearly O(n), which is acceptable.
\end{itemize}
\subparagraph{Weakness}
\begin{itemize}
	\item Can not know the meaning of each cluster, need further assessment. 
	\item The result and effect is difficult to assess.
	\item Need to set a appropriate parameter K value and find a suitable similarity measurement method.
\end{itemize}

\subsubsection{Model 2: KNN}
The full name of KNN is K-nearest neighbor. By using a similarity measurement method in \ref{sm}, for each user, KNN algorithm can recommend the top K daters who are well-matched with him or her. The distance smaller, the similarity degree bigger. To be clear, the goal here is not to classify a user to which class, but just to get the K-nearest neighbor as the Top K daters.
\paragraph{Few Options} The options here must have a characteristic that the difference value of a option between two users smaller, the matching degree of the two users higher. For example, the options could be goods and the values could represent how much users like the goods. Besides, personal information like educational background, age, height and weight can also be included. 

\paragraph{Inputs}  A numerical matrix with shape(u, onum): \textbf{User\_Item}. onum is the number of few options selected. Each value shows a user's evaluation or attribute about an option.  

\paragraph{Output} A two dimensional matrix with shape(u,2*K): \textbf{Top}. K columns store top K ID of matched daters and the other K columns store the scores. It could be a result of daters recommendation. By the way, it is a optimization compared to the \textbf{Scores} while \textbf{Top} needs fewer space.

\paragraph{Application}
Still based on data set \textbf{UCI Iris}, we run an example program using KNN algorithm. The full codes are shown in appendix \ref{apB}. We still use the Euclidean Distance as the measurement of the gap between two samples and assign the K parameter to 3. Since the KNN is a supervised learning process, we can calculate the accuracy and finally get an average accuracy rate 95.64\%. The results can be seen in figure \ref{1}.
\begin{figure}[h]
	%\small
	\centering
	\includegraphics[width=19cm]{m3.png}
	\caption{Results of Accuracy Rate with 300 run times}  \label{1}
\end{figure}
\newpage
\paragraph{Strength and Weakness}
\subparagraph{Strength}
\begin{itemize}
	\item Scientifically recommend the best k daters to each user, making it easy to choose friend from huge boundless sea of faces.
	\item It is a supervised learning process, which is very targeted and specific.
	
\end{itemize}
\subparagraph{Weakness}
\begin{itemize}
	\item Do not separate the affection from user A to user B and affection from user B to user A, which should be different since different person have different request, so constraints need to be introduced.
	\item With time complexity O(n*n), this algorithm may be a little slow.
	\item Options related needs to be filtered carefully by staff.
	
\end{itemize}

\subsubsection{Other Models }
Due to the limitation on time and data, we decide to just give a brief description of other models or algorithms that can be used in our daters recommendation system.

\begin{itemize}
	\item Collaborative Filtering Recommendation:
	Collaborative filtering is a method of making automatic predictions(filtering) about the interests of a user by collecting preferences or taste information from many users(collaborating). The underlying assumption of the collaborative filtering approach is that if a person A has the same opinion as a person B on an issue, A is more likely to have B's opinion on a different issue than that of a randomly chosen person. There are three mainly recommendation of collaborative filtering: Matrix factorization, Bayesian Belief Nets CF Models, Probabilistic Factor models.
	\item Content-based Recommendation
	\item Association Rule-based Recommendation
	\item Utility-based Recommendation
	\item Knowledge-based Recommendatio
	\item Hybrid Recommendation
\end{itemize}
%\subparagraph{Matrix factorization}
%
%\subparagraph{Bayesian Belief Nets CF Models}
%
%\subparagraph{Probabilistic Factor models}
%
%
%\paragraph{Content-based Recommendation}
%
%\paragraph{Association Rule-based Recommendation}
%
%\paragraph{Utility-based Recommendation}
%
%\paragraph{Knowledge-based Recommendation}
%
%\paragraph{Hybrid Recommendation}

\subsubsection{Models Optimization }

\begin{enumerate}
	\item Let's go back to the original established goal matrix \textbf{Scores}. When calculating the top K daters we just use comparison and Quick Sort Algorithm in each row of the matrix. For different objects, they may score differently from each other which means $Scores_{i,j} != Score_{j,i}$. So as for a user, we can add each $Score_{j,i}$ to each $Scores_{i,j}$ as a final matching degree, then do sorting algorithm on the sum matrix. In this way, the feelings of both sides are taken into account.
	\item Since before the K-means require one user only belong to one cluster, now we can divide all the options into many aspects and do K-means on every aspect. In this way, data sets are divided into many different hierarchical clusters according to different classification criteria and we know which cluster is relevant to which aspect. A user can be classified to more than one class and KNN algorithm can be done in every aspect, which means our recommendation is more specific and targeted.
	
\end{enumerate}

%\begin{figure}[!htbp]              
%	\centering
%	\subfloat[ACF and PACF]{                               
%		\includegraphics[width = .45\textwidth]{ACF.jpg}
%		\label{sub1}                                        
%	}
%	\qquad
%	\subfloat[Change in Russia`s Population]{                               
%		\includegraphics[width = .45\textwidth]{eluosi.jpg}
%		\label{sub2}                                                
%	}\\ 
%	
%	\caption{The Growth Model of Russia`s population}                                      
%	\label{eluosibijiaotu}                                          
%\end{figure}

%\begin{table}[H]
%	\centering
%	\caption{The Change of L1 Speakers in Four Countries}
%	\label{Atu}
%	\begin{tabular}{llllll}
%		\cline{1-3}
%		\multicolumn{1}{c}{}        & \multicolumn{1}{c}{Now} & \multicolumn{1}{c}{In ten years} &  &  &  \\ \cline{1-3}
%		\multicolumn{1}{c}{Russian} & \multicolumn{1}{c}{153} & \multicolumn{1}{c}{162}             &  &  &  \\
%		\multicolumn{1}{c}{English} & \multicolumn{1}{c}{371} & \multicolumn{1}{c}{405}             &  &  &  \\
%		\multicolumn{1}{c}{Chinese} & \multicolumn{1}{c}{897} & \multicolumn{1}{c}{955}             &  &  &  \\
%		\multicolumn{1}{c}{Jpanese} & \multicolumn{1}{c}{128} & \multicolumn{1}{c}{124}             &  &  &  \\ \cline{1-3}
%		
%	\end{tabular}
%\end{table}
%\begin{figure}[H]                                          
%	\centering
%	\includegraphics[width = .8\textwidth]{AtuL2.jpg}       
%	\caption{Model Test}                          
%	\label{AtuL2}                                          
%\end{figure}
%
%\begin{table}[H]
%	\centering
%	\caption{The Total Numbers of Four Languages}
%	\label{Bjieguo}
%	\begin{tabular}{cccclcl}
%		\toprule
%		& Native Language      &                      & \multicolumn{2}{c}{Second Language}                     & \multicolumn{2}{c}{Total}                               \\
%		\midrule
%		& Now                  & In Ten Years         & Now                  & \multicolumn{1}{c}{In Ten Years} & Now                  & \multicolumn{1}{c}{In Ten Years} \\
%		Russian              & 153                  & 162                  & 113                  & \multicolumn{1}{c}{119}          & 266                  & \multicolumn{1}{c}{281}          \\
%		English              & 371                  & 405                  & 611                  & \multicolumn{1}{c}{666}          & 982                  & \multicolumn{1}{c}{1071}         \\
%		Chinese              & 897                  & 955                  & 193                  & \multicolumn{1}{c}{205}          & 1090                 & \multicolumn{1}{c}{1160}         \\
%		Jpanese              & 128                  & 124                  & 1                    & \multicolumn{1}{c}{1.2}          & 129                  & \multicolumn{1}{c}{125.2} \\
%		\bottomrule
%	\end{tabular}
%\end{table}


%\begin{table}[H]
%	\centering
%	\caption{The Predictions of 6th-16th Language`s Total Numbers of Speakers}
%	\label{Bdaan}
%	\begin{tabular}{ccccccc}
%		\toprule
%		& \multicolumn{2}{c}{L1} & \multicolumn{2}{c}{L2} & \multicolumn{2}{c}{Total} \\
%		\midrule
%		& Now    & In 50 Years   & Now    & In 50 Years   & Now     & In 50 Years     \\
%		Malay      & 77     & 107           & 204    & 271           & 281     & 378             \\
%		Bengali    & 242    & 340           & 19     & 22            & 261     & 362             \\
%		Russian    & 153    & 183           & 113    & 158           & 267     & 341             \\
%		Portuguese & 218    & 297           & 11     & 16            & 229     & 313             \\
%		French     & 76     & 101           & 153    & 203           & 229     & 304             \\
%		Hausa      & 85     & 132           & 65     & 93            & 150     & 225             \\
%		Punjabi    & 148    & 192           &        &               & 148     & 192             \\
%		German     & 76     & 106           & 52     & 69            & 129     & 175             \\
%		Persian    & 60     & 84            & 61     & 85            & 121     & 169             \\
%		Japanese   & 128    & 164           & 1      & 1.3           & 129     & 165.3           \\
%		Swahili    & 16     & 22            & 91     & 131           & 107     & 153\\
%		\bottomrule
%	\end{tabular}
%\end{table}

\subsection{Solution for Problem Two}
By using Hybrid Algorithm (Take each result of recommendation algorithms into consideration), we can easily get the list of Top 20 Recommended Daters based on sorting the \textbf{Scores} matrix. For acquiring a more suitable estimate of an ideally sized choice set, we can make such an attempt: 
\begin{enumerate}
	\item Divide the users into 20 parts, for i\_th parts, we only recommend i daters, no more or less. 
	\item Gathering the results of dating and do evaluation about the recommendation effect which could be shown as date success rate and user satisfaction degree. 
	\item Do some analysis like regression and correlation on the number of daters and recommendation effect. Then it is clear to see the size of an ideal choice set for most people, which can be seen as large enough to include variety and depth and small enough that someone can fairly weigh each prospect’s potential without tripping his brain’s overload switch. 
	\item Analysis from the difference of character using the same way as step three, for diverse people we can recommend different size of choice set.
\end{enumerate}

%\begin{figure}[H]                                           
%	\centering
%	\includegraphics[width = .8\textwidth]{yiduixian.jpg}       
%	\caption{Global Population Mobility Network Space Pattern}                            
%	\label{yiduixian}                                         
%\end{figure}

%\begin{table}[H]
%	\centering
%	\caption{The Different Layer of The Forty Countries}
%	\label{ceng}
%	\begin{tabular}{ccccccclll}
%		\toprule
%		Hierarchy                & \multicolumn{5}{c}{Country}                                   \\
%		\midrule
%		The First Network Layer  & America      & Mexico      & Russian     & Ukraine  & India     \\
%		& Germany      & China       &             &          &          \\
%		\midrule
%		The Second Network Layer & Bangladesh   & Pakistan    & U.K         & France   & Canada    \\
%		& Italy        & Philippines & Iran        &          &           \\
%		\midrule
%		The Third Network Layer  & Turkey       & Spain       & Afghanistan & Algeria  & Poland   \\
%		& Morocco      & Japan       & Viet Nam    & Korea    &           \\
%		\midrule
%		The Forth Network Layer  & Brazil       & Colombia    & Argentina   & Iraq     & Congo   \\
%		& South Africa & Nigeria     & Thailand    & Tanzania & Myanmar   \\
%		& Indonesia    & Sudan       & Egypt       & Kenya    & Uganda   \\
%		& Ethiopia     &             &             &          &          \\
%		\bottomrule
%	\end{tabular}
%\end{table}

\subsection{Solution for Problem Three}
We design a form to collect the necessary information. We use this information to build user profiles and base on this information to recommend dating partner to users.\par 
This information contains five parts:
\begin{itemize}
	\item User's photo: The user must show his face in the photo.
	\item Basic information of the user: For instance, user's name or nickname and so on. See appendix \ref{ap2} for details.
	\item User social attribute information: For instance, user's family number and so on. See appendix \ref{ap2} for details.
	\item Information about ideal dating partner: For instance, the gender of dating partner and so on. See appendix \ref{ap2} for details. 
	\item Information about user character: We design a lot of multiple choice questions to test the user's personality. See appendix \ref{ap2} for details.
\end{itemize}
\par 
In life, when someone ask our what date partner we want to date, we usually use a lot of words to describe the date partner's character. For married people, personality compatibility between husband and wife has a great influence on the quality of marriage\cite{3}. So, we set up a lot of questions to analyze the user's personality in detail.Some domestic research shows that the similarities and differences between couples' character have on significant effect on the quality of marriage. But dissatisfaction with the character of the spouse is an important reason for the decline in the quality of marriage and even thee breakdown of marriage\cite{4}. So, we set up a lot of topics to analyze user preferences. We also set up some questions to get user's basic information, but this information is not the point.\par 
The answers to all questions on the questionnaire can be represented by numbers. So, we can build a matrix to describe the user. The user matrix will be our model's input.\par 
Forms are the basis for collecting information and the primary way to get user information. The quality of the form design has a great influence on the user's portrayal. Inaccurate descriptions of users will greatly affect the accuracy of the model. We create a regression prediction model to solve this problem.\par 
We have designed a total question bank. See appendix \ref{ap2} for details. The questions in the form are form this question bank. We design different forms by changing the number and type of questions in the form. At the same time, calculate the appointment success rate for people using different forms.\par 
We use a column vector to describe a form. Each value of the column vector corresponds to a test question. The total number of rows is the total number of questions in the test question bank. If question A appears in the form, the value which representative question A equal to 1. Else, the value equal to 0. Each form corresponds to a group of people who use the form. We calculate the success rate of dating in this group and think of it as a label for the form.\par
Finally, we group multiple column vectors which represent the forms into a matrix. This matrix is sample set and training data set. Every sample has a corresponding label. We use linear regression in supervised learning to fit the data. The prediction model is: \[f(x;b,w)=w^Tx+b.\] Parameters in prediction is $w \in \mathbb{R}^m$ and $b \in \mathbb{R}$. The input value $x$ is where $ x_j \in \mathbb{R}$ for $ j \in 1,...,m$. $m$ is the number of column vector's row.\par 
After training the model, first predict the label on the training set. After achieving higher accuracy on the training set, using the model, predict dating success rate on form which is not yet in use. At the same time put the form into use. Then, compare the difference between the predicted result and the true value and adjust the model to reduce the gap. When the accuracy of the model is stable at a relatively high level, the model is used to predict the newly designed un-used form, and the form design is adjusted based on the prediction result.\par 
Combine the form and appointment success rate through the model, plus the huge test data of the website, and finally, improve the appointment success rate through high-quality form design.

\subsection{Solution for Problem Four}
Using Data Mining to Match Dating Partner
Data-Mining Dating is a service within the online dating industry to use a scientific approach to matching highly compatible singles. 
Traditional Internet dating can be challening for those singles looking for love that lasts, but Data-Mining Dating is not a traditional dating site. Of all the single men or women you may meet online, very few will be compatible with you specifically, and it can be difficult to determine the level of compatibility of a potential partner through methods of conventional dating services. Our Predict System does the work for you  by narrowing the field from thousands of single prospects to match you with a select group of compatible matches with whom you can build a quality relationship. \par
Our algorithm aims to recommend perfect daters to each user from large crowds. With consideration of all factors you can imagine including personal requests, character matching degree, common outlook on world, life and values and so on, abundant algorithms based on model, content, rules are adopted. In the future, we even will take more factors such as face similarity analysis and gene matching degree into consideration. No best but better and better algorithms will be applied for recommending daters with higher matching degree to you. 

\section{Future Work}
The future work must be using a wider range of algorithms to do online recommendation tests, determining the best  similarity measurement method, and optimizing form design according to the form size-success rate relationship analysis.

\begin{thebibliography}{99}      
	\bibitem{1} Non-profit Organization.Loss function[DB/OL].(2018-10-02)[2018-12-02].https://en.  \\ wikipedia.org/wiki/Loss\_function    
	\bibitem{2} MuShi.Machine Learning: Comparisons of Several Distance Metrics[EB/OL].(2016-11-14)[2018-12-02].https://my.oschina.net/hunglish/blog/787596     
	 \bibitem{3} CHENG Zao-Huo, TAN Lin-Xiang, ZHAO Yong, et al. Spouse's Personality and Marital Quality[J]. Chinese Mental Health Journal, 2006, 20 (4): 268-271
	\bibitem{4} Li Ling-Jiang, Yang De-Sen. A control study on the personality of 100 couples in divorce proceedings[J]. Chinese Mental Health Journal, 1993, 007 (2):70-72
\end{thebibliography}

\begin{appendices}

	
	\section{Model 1: k-means -- example on iris data set \label{apA}}
    \begin{lstlisting}[language=python]	
	
	from sklearn.datasets import load_iris
	import matplotlib.pyplot as plt
	from numpy import *
	
	iris = load_iris() 
	data=iris['data']
	target=iris['target']
	X=data
	Y=target
	def calDistance(centroid, point):
	return sqrt(sum(power(centroid-point,2)))
	
	def constructCentroidSet(dataSet,K):
	numOfCoordinate = dataSet.shape[1]
	
	centroidSet=mat(zeros((K,numOfCoordinate)))

	for ith_coordinate in range(numOfCoordinate):  
	min_ith_coordinate=min(dataSet[:,ith_coordinate])
	max_ith_coordinate=max(dataSet[:,ith_coordinate])
	range_coordinate=max_ith_coordinate-min_ith_coordinate  
	centroidSet[:,ith_coordinate] = min_ith_coordinate+range_coordinate* \
	random.rand(K,1)
	return centroidSet
	
	def kMeans(dataSet, k):
	numOfsamples= dataSet.shape[0]  
	
	class_distance = mat(zeros((numOfsamples,2)))   
	centroidSet = constructCentroidSet(dataSet, k)
	NoChangeHappened = False  
	
	while not NoChangeHappened:
	NoChangeHappened = True;
	for ith_sample in range(numOfsamples): 
	minDistance = inf 
	classIndex = -1     
	for jth_cluster in range(k):
	distance_ith_sample_jth_cluster = calDistance(centroidSet[jth_cluster,:], \
	 dataSet[ith_sample,:])

	if distance_ith_sample_jth_cluster < minDistance:
	minDistance = distance_ith_sample_jth_cluster
	classIndex = jth_cluster
	
	if class_distance[ith_sample,0] != classIndex: 
	NoChangeHappened = False  
	class_distance[ith_sample,:] = classIndex , minDistance 
	
	for ith_centroid in range(k):   
	class_row_isCentroid= nonzero(class_distance[:,0].A==ith_centroid)[0]      
	sample_is_centroid = dataSet[class_row_isCentroid]   
	
	centroidSet[ith_centroid,:] = mean(sample_is_centroid, axis = 0)  
	#print(centroidSet)
	
	return centroidSet, class_distance
	

	def getClassValue(aclass):
	for i in aclass:
	return argmax(bincount(i))

	
	accuracies=[]
	run_times=300 
	for test_times in range(run_times):
	
	centroids,class_distance=kMeans(data,3)
	#print(class_distance)
	#print(centroidSet)
	
	
	class1=(array(class_distance[0:50,0]).T).astype(int)
	class2=(array(class_distance[50:100,0]).T).astype(int)
	class3=(array(class_distance[100:150,0]).T).astype(int)

	class1_name=getClassValue(class1)
	class2_name=getClassValue(class2) 
	class3_name=getClassValue(class3) 
	#print('class1: ',class1_name,'class2: ',class2_name,'class3: ',class3_name)
	
	accuracy=0
	accuracy= (sum(class1==class1_name)+sum(class2==class2_name)+ \
	sum(class3==class3_name))/150
	print(accuracy)
	accuracies.append(accuracy)
	print("average accuracy: ")
	print(sum(accuracies)/run_times)

	
	%matplotlib inline
	import matplotlib.pyplot as plt
	
	plt.figure(figsize=(18,10))
	plt.plot(accuracies, "-", color="black", label="accuracy")
	plt.xlabel("run_times")
	plt.ylabel("accuracy")
	plt.legend() 
	plt.title("kMeans")
	
	\end{lstlisting}
    \section{Model 2: KNN -- example on iris data set \label{apB}}
    \begin{lstlisting}[language=python]
    import pandas as pd
    import numpy as np
    
    class kNN:
    def __init__(self,X,y,split=0.2,test='YES'):
    
    if isinstance(X,pd.core.frame.DataFrame) != True:  
    self.X = pd.DataFrame(X)
    else:
    self.X = X
    if isinstance(y,pd.core.series.Series) != True:
    self.y = pd.Series(y)
    else:
    self.y = y  
    
    self.max_data = np.max(self.X,axis=0)
    self.min_data = np.min(self.X,axis=0)
    max_set = np.zeros_like(self.X); max_set[:] = self.max_data 
    min_set = np.zeros_like(self.X); min_set[:] = self.min_data
    self.X = (self.X - min_set)/(max_set - min_set)
    
  
    if test == 'YES':     
    self.test = 'YES'   
    n_samples = len(self.X)
    trainDataSet = [i for i in range(n_samples)]  # 0-149
    testSet = []                          
    for i in range(int(n_samples*split)):
    random_num = trainDataSet[int(np.random.uniform(0,len(trainDataSet)))]
    testSet.append(random_num) 
    trainDataSet.remove(random_num)
    self.X,self.testSet_X = self.X.iloc[trainDataSet],self.X.iloc[testSet]
    self.y,self.testSet_y = self.y.iloc[trainDataSet],self.y.iloc[testSet]
    else:
    self.test = 'NO'
    
    def getDistances(self,point):  
    points = np.zeros_like(self.X)   
    points[:] = point                
    minusSquare = (self.X - points)**2  
    EuclideanDistances = np.sqrt(minusSquare.sum(axis=1)) 
    return EuclideanDistances
    
    
    def getClass(self,point,k):    
    distances = self.getDistances(point)
    argsort = distances.argsort(axis=0)       
    classList = list(self.y.iloc[argsort[0:k]])
    classCount = {}
  
    for i in classList:
    if i not in classCount:
    classCount[i] = 1
    else:
    classCount[i] += 1
    maxCount = 0
    maxkey = 'x'
    for key in classCount.keys():
    if classCount[key] > maxCount:
    maxCount = classCount[key]
    maxkey = key
    return maxkey
    
    
    def knn(self,testData,k):     
    if self.test == 'NO':    
    testData = pd.DataFrame(testData)
    max_set = np.zeros_like(testData); max_set[:] = self.max_data
    min_set = np.zeros_like(testData); min_set[:] = self.min_data
    testData = (testData - min_set)/(max_set - min_set)   
    if testData.shape == (len(testData),1): 
    label = self.getClass(testData.iloc[0],k)
    return label                      
    else:
    labels = []
    for i in range(len(testData)):
    point = testData.iloc[i,:]
    label = self.getClass(point,k)
    labels.append(label)
    return labels                    
    
    
    def errorRate(self,knn_class,real_class):   
    error = 0
    allCount = len(real_class)
    real_class = list(real_class)
    for i in range(allCount):
    if knn_class[i] != real_class[i]:
    error += 1
    return error/allCount
    \end{lstlisting}

	\section{Information Forms Design \label{ap2}}
		
		\begin{itemize}
		\item User basic information:
		\begin{enumerate}
			\item name/nickname
			\item I am a man/woman.
			\item How many children do you have?
			\item When were you born?
			\item Where do you live?
			\item What is your nation?
			\begin{itemize}
				\item white
				\item Hispanic/Latino
				\item black/African descent
				\item Asian/pacific islander
				\item Indian
				\item Chinese			
				\item native American
				\item Arabic/middle eastern
				\item Korean
				\item Japanese
				\item other
			\end{itemize}
			\item What best describes your religious beliefs or spirituality?
			\begin{itemize}
				\item christian
				\item Jewish
				\item Muslim
				\item Hindu
				\item Buddhist
				\item Sikh
				\item Shinto
				\item other
				\item spiritual
				\item but not religious
				\item neither religious nor spiritual
				\item Baha'i
				\item cao dai
				\item Confucianism
				\item Jainism
				\item christian science
				\item Rastafarianism
				\item Taoism
				\item Tokyoite
				\item Unitarian-universalism
				\item Scientology
				\item metaphysical
				\item pagan
				\item Wiccan
				\item new age
				\item prefer not to specify 
			\end{itemize}  
			\item Which describes your highest level of education?
			\begin{itemize}
				\item doctorate
				\item masters
				\item bachelors
				\item associates
				\item some college
				\item high school
			\end{itemize} 
			\item What is you job?
			\item What's your personal income?(Your matches won't see this.)
			\item How often do you smoke?
			\begin{itemize}
				\item never
				\item socially
				\item once a week
				\item few times a week
				\item daily
			\end{itemize}
			\item How often do you drink?
			\begin{itemize}
				\item never
				\item on special occasions
				\item once a week
				\item few times a week
				\item daily
			\end{itemize}
			\item How tall are you?
			\item What are you passionate about?
			\item What two or three things do you enjoy doing with you leisure time?
		\end{enumerate}
		\item User social attribute information.(Optional)
		\begin{enumerate}
			\item Family members
			\item Graduated school
		\end{enumerate}
		\item Information about ideal dating partner.
		\begin{enumerate}
			\item I am seeking a man/woman.
			\item I am looking for someone between the ages of xx-xx.
			\item How far should we search for your matches?  x miles.
			\item How important is the distance of your match?
			\begin{itemize}
				\item not at all important
				\item somewhat important
				\item very important
			\end{itemize}
		\end{enumerate}
		\item Information about user character.
		\begin{enumerate}
			\item How well dose this generally describe you?(Not at all / somewhat / very well)
			\begin{itemize}
				\item warm
				\item clever
				\item dominant
				\item outgoing
				\item quarrelsome
				\item stable
				\item energetic
				\item predictable
				\item affectionate
				\item intelligent
				\item attractive
				\item compassionate
				\item loyal
				\item witty
				\item sensitive
				\item generous
				\item sensual
				\item stylish			
				\item athletic
				\item overweight
				\item plain
				\item healthy
				\item sexy
				\item content
				\item patient
				\item passionate
				\item caring
				\item genuine
				\item vivacious
				\item wise
				\item bossy
				\item leader
				\item irritable
				\item kind
				\item aggressive
				\item outspoken
				\item opinionated
				\item restless
				\item romantic
				\item selfish
				\item stubborn
				\item I do things according to a plan.
				\item I take time out for others.
				\item I feel unable to deal with things.
				\item I love to help others.
				\item I seek adventure.
				\item I desire sexual activity.
				\item I often leave a mess in my room.
				\item I often carry the conversation to a higher level.
				\item I get stressed out easily.
				\item I often make others feel good.
				\item I am good at analyzing problems.
				\item I usually stand up for myself.
				\item I am easily discouraged.
				\item I can handle a lot of information.
				\item I waste my time.
				\item I catch on quickly.
				\item I usually wait for others to lead the way.
				\item I love order and regularity.
				\item I often do nice things for people.
				\item I get angry easily.
				\item My personal religious beliefs are important.
				\item I ask questions in search of information.
				\item I think it is important to continually try to improve myself.
				\item I care about the physical shape I'm in.
				\item I feel better when I am around other people.
				\item I try to accommodate the other person's position.
				\item I try to understand the other person.
				\item I try to be respectful of all opinions different from my own.
				\item I try to resolve conflict well.
			\end{itemize}
			\item How strongly do you agree or disagree with...?(Absolutely disagree / Neither agree nor disagree / Absolutely agree)
			\begin{itemize}
				\item I am looking for a long-term relationship that will ultimately lead to marriage.
				\item When I get romantically involved, I tell my partner just about everything.
				\item It is difficult for me to let people get emotionally close to me.
				\item A "serious" relationship needs to be exclusive (i.e. monogamous).
				\item I know I can always count on the people who are closest to me.
				\item I don't need to have close relationships to be happy.
				\item Being monogamous helps build intimacy and trust in a romantic relationship.
				\item People often let you down if you depend on them.
				\item It's important for me to have close friends in my life.
				\item Being exclusive (i.e., monogamous) is one of benefits of being in a successful relationship.
				\item I sometimes find it difficult to trust people I get romantically involved with.
				\item I find it easy to get emotionally close to people.		
			\end{itemize}
			\item How important in a relationship is...(Not at all important / Somewhat important / Very important)
			\begin{itemize}
				\item My partner's dependability.
				\item My partner's sex appeal.
				\item My partner's physical appearance.
				\item Enjoying the way I feel around my partner.
				\item Our sexual compatibility.
				\item The friendship between me and my partner.
				\item Enjoying physical closeness with my partner.
				\item Being able to spend as much time as possible with my partner.
				\item Doing special things to let my partner know how important he/she is to me.
			\end{itemize}
			\item How happy are you with your physical appearance?
			\begin{itemize}
				\item Not at all
				\item Somewhat happy
				\item Very happy
			\end{itemize}
			\item How often in the past month have you felt...?(Really / Occasionally / Almost Always)
			\begin{itemize}
				\item Happy
				\item Sad
				\item Anxious
				\item Confident
				\item Hopeful
				\item Fearful about future
				\item Angry
				\item Calm
				\item Fortunate
				\item Out of control
				\item Fulfilled
				\item Depressed
				\item Unable to cope
				\item Satisfied
				\item Misunderstood
				\item Plotted against			
			\end{itemize}
			\item How skilled you are at the following things:(Not skilled / Somewhat Skilled / Very Skilled)
			\begin{itemize}
				\item Creating romance in a relationship
				\item Keeping physically fit
				\item Finding and taking on challenging activities
			\end{itemize}
			\item What's your interest in....?(None / Some Interest / Very Strong Interest)
			\begin{itemize}
				\item Watching movies
				\item Listening to music
				\item Watching TV
				\item Reading
				\item Parties
				\item Dining out
				\item Traveling
				\item Shopping
				\item Family
				\item Talking with friends
				\item Religious Community
				\item Religious Faith
				\item Conversation
				\item Hosting/Entertaining
				\item Church Involvement
			\end{itemize}
			\item If your best friends had to pick four words to describe you, which four from this list would they pick?
			\begin{itemize}
				\item good listener 
				\item modest
				\item respectful
				\item affectionate
				\item caring
				\item spontaneous
				\item physically
				\item fit
				\item warm
				\item outgoing
				\item optimistic
				\item dependable
				\item romantic
				\item creative
				\item loyal
				\item spiritual
				\item kind
				\item ambitious
				\item articulate
				\item rational
				\item easy-going
				\item generous
				\item happy
				\item quiet
				\item genuine
				\item intelligent
				\item sweet
				\item passionate
				\item energetic
				\item funny
				\item perceptive
			\end{itemize}
		\end{enumerate}
	\end{itemize}

\end{appendices}
                      
    \section{Introduction}
    \subsection{Background}                       
	In 2017, the worst hurricane to ever hit the United States territory of Puerto Rico, left the island   severe damage and caused over 2900 fatalities. The combined destructive power of the hurricane's storm surge and wave action produced severe damage to homes, buildings, utility poles, transmission lines, cellular networks, highways and roads. Demands for medical supplies and lifesaving equipment zoomed up while the full extent of the damage remained unclear and dozens of areas were isolated.\par
	Under this background, a non-governmental organizations - HELP, Inc. - is attempting to
	improve its response capabilities by designing a transportable disaster response system called "DroneGo", which is a prologue to this paper. 
	

    \subsection{Clarification of the Problem}             
  We use the 2017 situation in Puerto Rico to design the DroneGo. Four problems wait us to be solved:
    \begin{enumerate}       
    	\item \textbf{Packing configuration of cargo containers:\label{P1}} Recommend a drone fleet and set of medical packages and design the associated packing configuration for each of up to three ISO cargo containers. We have to create a optimization model for the selection scheme since the options are various.
    	\item \textbf{Location of cargo containers:\label{P2}} Identify the best location or locations on Puerto Rico to position one, two, or three cargo containers of the DroneGo. We have to analysis the pros and cons of different quantity of containers and different positions and finally determine the best plan.
    	\item \textbf{Delivery plan:\label{P3}} Design the associated payload packing configurations of the drone fleet, then produce the delivery routes and schedule to meet the identified medical package requirements of the hurricane scenario. We have to choose a comprehensive scheme for both consideration of resources and efficiency.
    	\item \textbf{Reconnaissance plan:\label{P4}} Provide a drone flight plan, enable the DroneGo fleet to use cameras to assess the major highways and roads. We have to create a scientific model to cover as much area of Puerto Rico as possible.
    \end{enumerate}
    
%    
%    \subsection{Our Work}                      
%    In order to make the prediction more accurate, our work is divided into the following four aspects:

    
     \section{Preparation of the Models}
    \subsection{Assumptions}
    1. Neglect small islands around Puerto Rico; \label{as1}
    
    2. Cargo containers are kept in safe places and will not be damaged by disasters. \label{as2}
    
    3. The angle between the visual field of drone's camera and the vertical direction is 60 degrees, the flight altitude of drone is 800 meters and the lines of sight are always straight.  \label{as3}
    
    4. Consider \textbf{three day}'s medical supply delivery tasks and the reconnaissance task only needs to be done once.\label{as4}
    
    5. Because of the loss of power, we \textbf{do not consider the charging of drones}, which means the drones are disposable and we need to prepare three drone fleets for the delivery task.  \label{as5}
    
    6. Assume that if a drone is near a hospital, the drone can land in the hospital and needn't go back. Otherwise, drones must have the capability to come back to the container. \label{as6}
    
    7. The drone's maximum journey distance changes to half when it's full loaded and the affect of payloads on max speed and max flight time are uniform. \label{as7}
    
    8. H Drone is used for communication between the drone fleet and HELP, Inc.'s command and control center.
    \subsection{Variable Declaration} 
    \begin{center}
    	\begin{tabular}{c|l}
    	\toprule[2pt]
    \label{s}
    		\makebox[0.15\textwidth][c]{\textbf{Symbol}}	& \makebox[0.1\textwidth][c]{\textbf{\qquad\qquad\qquad\qquad\qquad\qquad\qquad Definition }}    \\ 	\midrule[1pt]
    		v & Speed of a drone \\ 
    		$t_{max}$ & Maximum flight time no cargo of a drone \\  
    		s  & The maximum journey distance of a drone  \\
    		s1 & The distance from a cargo container to the grid designated \\ 
    		s2 & The distance of traveling a whole small grid \\ 
    		s3 & The distance from the grid to the   nearby hospital \\ 
    		Cr & The coverage rate of grids on Puerto Rico \\ 
    		H  & The flight altitude of drones  \\  
    		R  & The width of visual field of drones' cameras \\
    		S  & The area of each grid \\
    		tt & The maximum time available for a drone to traverse\\
    		&  in the target grid \\
    		Crg & The traversal coverage rate of a drone in a grid \\
    	    $\theta$ & The angle between the visual field of drone's camera and \\
    	    & the vertical direction \\
    		$G(m,n)$ & All grids consist of m rows and n columns \\
    		$N$ & Number of grids (equals to $m*n$) \\
    		$C$  & The number of grids that are covered in our solution\\
    		$THD$ &   The matrix of traffic highway density, which also has \\
    		& m rows and n columns     \\
    		$PD$  &   The matrix of population density, which also has\\
    		&  m rows and n columns \\
    		$mpc$ &   Max payload capability of a drone \\
	$p$   &   The actual payload of a drone \\
$V$   &   The volume of a drone \\
$V'$  &   An index for rating capacity of drones that is related to $V$ \\
$l,w,h$   &   Length, width and height of drones   \\
$bt$  &  The drone cargo bay type \\
    	    $bt'$ &  An index for rating capacity of drones that is related to $bt$  \\
    	\bottomrule[2.5pt]
    	\end{tabular}
    \end{center}

\begin{center}
	\begin{tabular}{c|l}
		\toprule[2pt]
		\makebox[0.15\textwidth][c]{\textbf{Symbol}}	& \makebox[0.1\textwidth][c]{\textbf{\qquad\qquad\qquad\qquad\qquad\qquad\qquad Definition }}    \\ 	\midrule[1pt]
    $\alpha$ & Coverage rate of grids number  \\
    $\beta$ & Coverage rate of grids weights \\
    	\bottomrule[2.5pt]
\end{tabular}
\end{center}
    \section{Grids Traversing Model (GTM)}                                    
    The DroneGo is made up of two parts - delivery system and reconnaissance system - both of which can be finished by drones. By doing some estimate, we find the daily delivery task can be easily satisfied while the reconnaissance task are hard to achieve. So we propose Grids Traversing Model firstly, which are aimed at \textbf{reconnoitring} as much area as possible and determining the optimum \textbf{location}.
\subsection{Grids Partition}
 Obviously, the maximum stroke of a drone is very limited (less than 52.66km). To reconnoitre more area, we can set the scanning task of each drone in one appropriate grid. Each drone mainly needs to reconnoitre several grids (of course plus the route it travels) and comes back so that it's convenient to make the flight plan.
    \subsection{Determine the Grid Size}  
    \subsubsection{Grid Constraints} 
    The grid is supposed to meet the following constraints:
    	\begin{enumerate}
    		\item The size must be appropriate for the drone to finish traversing task in limited time.  
    		\item Grids must be regular graphics, better be rectangle or even square, which are easily to be partitioned and traversed. 
    		\item All grids have the same size and no overlap exists.
    		\item Grids must cover every area as much as possible, and the coverage rate $Cr$ must be greater than 90\%. 
    		
    	\end{enumerate} 
    \par
   	 Besides, to determine the size of a grid, we have to calculate the width of drone's field of vision. We propose the next model to quantify it.
  
   	\subsubsection{Simple Cone Model }
   	Assume cameras of drones face down, then the surface surrounded by visual field and the ground make up a cone. Figure \ref{1} shows the detail\cite{1}.
   	 \begin{figure}[H]                                         
   		\centering
   		\includegraphics[width = .2\textwidth]{capture.PNG}      
   		\caption{ A Drone's Fight View }                            
   		\label{1}   
   	\end{figure} 
   	The diameter of bottom circle determines how big area the drone can reconnoitre in limited time. If a drone has time $tt$ to traverse in the grid, then its maximum traversal area is $2*R*v*tt$. The $R$ is equal to $H*tan\theta $. Then we have the following formula for calculating the traversal coverage rate in a grid:
   	\begin{equation} 
   			Crg=2R*v*tt/S 
   	\end{equation}
   	To get bigger $Crg$, smaller $S$ is better, but smaller $S$ means there are more grids, which may increase the complexity of fight plan. 
   	In addition, We don't take it into consideration that the turning distance may cause loss and the drone may need to come back to the place entering the grid from the place traversing all the grid. Besides, the traversing route is S-shaped.\par
   	According assumption 3, the flight altitude is 800 m and $\theta$ is 60 degrees, which determine the width of field of view is $2*800* tan 60^\circ \approx 2770 m.$ 
   	After some comprehensive calculation and bold estimate, we determine to construct $12*35=420$ square grids whose edge length is 5 km. Figure \ref{gp} shows the specific segmentation.
   	    \begin{figure}[H]                                         
   		\centering
   		\includegraphics[width = .8\textwidth]{grids.jpg}      
   		\caption{Grids Partition (12 rows * 35 columns) }                            
   		\label{gp}                                          
   	\end{figure}
   
   \subsection{Grids Weights Assignment}
   The partitioned grids are not only beneficial for providing a drone flight plan(Problem \ref{P4}), but also for the resolution of optimum location(Problem \ref{P2}). Since the number and capacity of drones are not enough to traverse the whole Puerto Rico Area, we can set a weight to each grid. In this way, we can give priority to grids with high weights and locate the cargo contain where most high weighted grids can be reconnoitred.\par
   We consider two factors to influence the weights of grids: traffic highway density and population density. After that we use entropy weighting model(see section \ref{EWM}) to take the two factors into account and generate the final weights form(or matrix).
   \paragraph{Traffic Highway Density }
   The highway density in a grid can indicate how important the grid is deserved to be reconnoitred. We can use formula \ref{hh} to calculate each grids' density\cite{2}:
   \begin{equation}  \label{hh}
   V=\sum\limits_{i=1}^4{Q_i}= \sum\limits_{i=1}^4 {C_i\alpha_i\beta_i\gamma_iN_iL_i}
   \end{equation}
   In this formula, $V$ means the total throughput of road network(pcu
   ${{\cdot}}$km/h), which can represent the traffic highway density. $Q_i$ means the throughput of class i road. i equals to 1 $\sim$ 4 which represent expressway, main road, secondary road and Access Rd. $C_i$ means accessible capacity of single lane on class i road. $\alpha_i$ means the saturation of class i road. $\beta_i$ means reduction coefficient at intersection of class i road. $\gamma_i$ means the lane comprehensive reduction coefficient of class i road. $N_i$ means the average number of motor vehicle lanes of class i road. $L_i$ means the mileage of class i road(km).  
   
   Due to time and data limitation, we grade each grid with a rank (0$\sim$3) according to the above factors as figure \ref{thd} shows. Grade 0 represents it's low density while grade 3 means high density. Figure \ref{tdm} shows the concentration of traffic density.
	   \begin{figure}[!htbp]                               
		\centering
		\subfloat[Traffic Density Grade Evaluation ]{                             
			\includegraphics[width = .45\textwidth]{jt2.jpg}
			\label{thd}  		                                        
		}
		\qquad
		\subfloat[Traffic Density Map ]{                               
			\includegraphics[width = .45\textwidth]{jt.png}  
			\label{tdm}                                          
		}\\ 
	\caption{Traffic Highway Density Grade and Concentration Picture}                             
\end{figure}
    \paragraph{Population Density}
    We can find some population data of Puerto Rico's 78 cities and the total population data over the years in \cite{3} and \cite{4}. By regional population underestimate, we can also obtain the population matrix containing each grids' population value. The figure \ref{cmdp} shows our result. \par
    
       \begin{figure}[!htbp]                                         
    	\centering
    	\includegraphics[width = .8\textwidth]{rk.png}      
    	\caption{ Contour map and Density of Population }                            
    	\label{cmdp}                                          
    \end{figure}
    So far, we get two density map which is helpful for the selection of optimum container location. Two matrix $THD$ and $PD$ are built, which can be the input of next model - \textbf{Entropy weighting model(EWM)}(see section \ref{EWM}) - and finally we obtain that the population density to traffic highway density ratio is \textbf{0.6172 to 0.3828} and produce the grids weights matrix. 
     We use \textbf{python} to calculate the whole calculation process and use \textbf{matlab} to carry out visualization. Figure \ref{finalweights} is part of our result. For making the gap more obvious, we multiply each value by 1000 thus the sum of weights is equal to 1000. For some areas which are inaccessible such as high mountains, we set the weights of these grids to zero. By the way, the three concentration maps will benefit for containers location while the grids weights matrix will benefit for making reconnaissance plan. 
    	   \begin{figure}[!htbp]                               
    	\centering
    	\subfloat[A Part of the Grids Weights Matrix ]{                             
    		\includegraphics[width = .45\textwidth]{finalweighs.png}
    		\label{finalweights}  		                                        
    	}
    	\qquad
    	\subfloat[Concentration of Grids Weights]{                               
    		\includegraphics[width = .45\textwidth]{fw.PNG}  
    		\label{thd2}                                          
    	}\\ 
    	\caption{Traffic Highway Density Grade and Concentration Picture}                             
    \end{figure}

    \section{Entropy Weighting Model (EWM)} 
    \label{EWM}
      Entropy Weighting Model is aimed at calculating weights. For a matrix with i rows and j columns(i means the number of samples and j means the evaluation index quantity), we want to give a weight to every evaluation index and rank all the samples. In DroneGo, we have two evaluation index: traffic density and population density, which means j equals to 2. And i can mean the number of grids($m*n$).
    There are mainly three steps in this model: normalization processing, entropy calculation, weights and ranks generation \cite{5}.

   \subsection{Normalization Processing}
   We use the followed formula to do standardization:
   \begin{equation}
   {{\rm{R}}_{i,j}} = \frac{{{R_{i,j}} - {{{\mathop{\rm minR}\nolimits} }_j}}}{{\max {R_j} - \min {R_j}}}
   \end{equation}
   R means the raw matrix, which consists of $THD$ and $PD$. But we need to transfer $THD$ and $PD$ to two column vectors. $R_j$ mean a column vector($THD$ or $PD$). "max" and "min" mean the maximum and minimum value of the column vector. To achieve normalization, the sum of a column vector must be equal to 1. So we only need to divide each value by the sum of the column it belongs to.
     
   \subsection{Entropy Calculation}
   Use $E_j$ expresses the total contribution of all samples to the evaluation $j$, which is calculated by the formula \ref{ej}.
   \begin{equation}
   \label{ej}
   {E_j} =  - K\sum\limits_{i = 1}^N {{R_{i,j}}\ln {R_{i,j}}} 
   \end{equation}
   $N$ means the number of grids($m*n$). $K = \frac{1}{{\ln (N)}}$, which is to ensure that $E_j$ belongs to 0$\sim$1. When the contribution value in $R_j$ tend to be consistent, $E_j$ is close to 1. Otherwise, $E_j$ is close to 0. 
     \subsection{Weights and Ranks Generation}
   Use formula \ref{dw} to calculate the weights. $J$ means the number of evaluation indexes.
   \begin{equation}
   {W_j} = \frac{{1 - {E_j}}}{{\sum\limits_{j = 1}^J {(1 - {E_j})} }} \label{dw}
   \end{equation}
   Finally, we can calculate the score of each sample(each grid) and rank them if necessarily.
   Take traffic highway density and population density as an example, the whole flow of $EWM$ is shown in Figure \ref{liucheng}.
      \begin{figure}[!htbp]                                         
    	\centering
    	\includegraphics[width = .9\textwidth]{liucheng.PNG}      
    	\caption{ EWM Flow  }                            
    	\label{liucheng}                                          
    \end{figure}
  
  \section{Resource Allocation Model (RAM)}
  For the purpose of recommending a reasonable drone fleet, set of medical packages and packing configuration(Problem \ref{P1}), we can rate each type of drone firstly with the help of \textbf{EWM}, which can obtain the comprehensive capability of drones in medical supply delivery and video reconnaissance. Then we use a set of heuristic algorithms to solve the 3D Bin Packing Problem. Please note that here we choose \textbf{3 cargo containers} to calculate directly and already know the positions of containers. The detailed reason is shown in section \ref{ols}.
  
  \subsection{Drones Rating Method}
  \label{ldis}
  The \textbf{flying range of drones} is our first focus. We think that the range is the most important index when we choose drones. According to assumption 7, we have the following equation to calculate the maximum loaded flying range:
  \begin{align}
  \label{8}
  {v_{loaded}} &= \sqrt {1 - \frac{1}{2}\frac{p}{{mpc}}} v\\
  {t_{loaded}} &= \sqrt {1 - \frac{1}{2}\frac{p}{{mpc}}} {t_{\max }}\\
  s &= {v_{loaded}}*{t_{loaded}}
    \label{dc}
  \end{align}
%  \begin{equation} 
%  s = \sqrt {1 - \frac{1}{2}\frac{p}{{mpc}}} v*\sqrt {1 - \frac{1}{2}\frac{p}{{mpc}}} {t_{\max }}
%  \end{equation}
   In the rating process, we use the average weight $p$ = 6 pounds of the medical kits that each hospital needs every day to calculate $s$ with payloads. Since the max payload capability of A Drone is 3.5 (less than 6), we set its medical supply task capacity rate to 0.
   
   The dimension of drones is an important index. We can use $V=l*w*h$ to describe the size of drones. The longest edge of drones is also a meaningful variable. But there is obvious difference between it and $V$, we ignore it in rating method. In order to make full use of the container space, it's better if the size of drones for reconnaissance is smaller. Hence we use \textbf{the reciprocal of $\mathbf{V}$} as a rating index. Furthermore, the changing range of rates is an important variable affecting the weights in the subsequent information entropy calculation, so we control the changing range of rates in $V$ by \textbf{power reduction operation}, which makes the changing range of rates in $V$ equals to that in $s$. The final formula using data $V$ is:
   \begin{equation}
   V' = {(\frac{1}{V})^{\frac{2}{3}}} = {(\frac{1}{{l + w + h}})^{\frac{2}{3}}}
   \end{equation}
   For the configuration capabilities indexes, if drones do not have the ability to do a task, then the rate of this kind of drones in this task index is 0. Particularly for \textbf{bay type} index $bt$, we increase its changing range by increasing the power in like manner as operation on $V$: $ \mathbf{bt'}=bt*bt$. \par
   With the description above, we generate a normalized drones attribute matrix with four indexes shown in Figure \ref{tdge}. 
      \begin{figure}[!htbp]                               
   	\centering
   	\subfloat[Attributes of Drones on Four Indexes]{                             
   		\includegraphics[width = .45\textwidth]{rate.png}
   		\label{tdge}  		                                        
   	}
   	\qquad
   	\subfloat[Final Scores of Drones on Two Tasks ]{                               
   		\includegraphics[width = .45\textwidth]{finalRate.png}  
   		\label{fsd}                                          
   	}\\ 
   	\caption{Attribute Matrix and Rating Results} 
\end{figure}
\par
	Now we can calculate the weights of indexes under different tasks with the help of \textbf{EWM}.  
	\textbf{For delivery task}, we consider three dimensions:  $s$ with average payload, $V'$ and $bt'$. Use these three columns of data in Figure \ref{tdge} as the input of EWM to calculate the weights of the three index. The result is: weight\_$s$(0.4130), weight\_$V'$(0.2470), weight\_$bt'$(0.3400).
	\textbf{For reconnaissance task}, we consider two dimensions: $s$ with average payload, $V'$ ($V'$ is necessary since it can affect how many drones the cargo contains can accommodate). Use the two columns from Figure \ref{tdge} as the input of EWM to calculate the weights of the two indexes. The result is: weight\_$s$(0.6879), weight\_$V'$(0.3121). These results verify that our conjecture that maximum journey distance $s$ is the most influential variable. \par
	Then use the following two equations to calculate the final scores of each type of drone on two tasks.
	\begin{align}
	Score_{delivery}&=weight\_s * s_{with\_payload} + weight\_V' * V' + weight\_bt' * bt' \\
	Score_{reconnaissance}&=weight\_s * s_{with\_no\_payload} + weight\_V' * V'
	\end{align}
%	\begin{equation}
%	Score_{reconnaissance}=weight\_s * s_{with\_no\_payload} + weight\_V' * V'
%	\end{equation}
	The final scores of each drone is shown in Figure \ref{fsd}. It's clear to see that \textbf{F Drone} is best suited for delivery task and \textbf{B Drone} is best suited for reconnaissance task.
	
	\subsection {Drones and Containers  Configuration for Packing}
	From the result of drones rating method above, we decide to pack as many B drones as possible with the prerequisite that there are three sets of medical packages and drone fleets that could ensure the resource available for three day's delivery task (Assumption 4). From the analysis of OLM (section \ref{ols}), we obtain the position of three containers. Hence it's easy to determine which container responsible for delivery of corresponding hospitals, and then determine which types of drones, how many drones to do delivery tasks. As for the routes and schedule, we expect all the drones take the air (shortest) line to hospital. Then we can calculate the fight time according to equation \ref{8} to \ref{dc}. The drones' payload packing configuration and schedule are shown in Table \ref{dcp}. 
	\begin{table}[H]
		\centering
		
		\begin{tabular}{lllcc}
			\cline{1-5}
			
			\multicolumn{1}{c}{Hospital (abbr.)}  & \multicolumn{1}{c}{Container}         & \multicolumn{1}{c}{Drone} & \multicolumn{1}{c}{Distance (km)}     & \multicolumn{1}{c}{Fight Time (hour)} \\
			\cline{1-5}
			
			\multicolumn{1}{c}{CMC}  & \multicolumn{1}{c}{3}  &\multicolumn{1}{c}{B: 1*3}    & 36 & 0.55   \\
			\multicolumn{1}{c}{HH}   & \multicolumn{1}{c}{3}  &\multicolumn{1}{c}{E: 1*3}    & 7 &  0.14  \\
			\multicolumn{1}{c}{HPS}  & \multicolumn{1}{c}{3}  &\multicolumn{1}{c}{B: 1*3}  & 27 & 0.39 \\ 
 			\multicolumn{1}{c}{PRCH} & \multicolumn{1}{c}{3}  &\multicolumn{1}{c}{B: 1*3;  C: 1*3}  & 28 & B: 0.448; C: 0.494     \\
 			\multicolumn{1}{c}{HPA}  & \multicolumn{1}{c}{1}  &\multicolumn{1}{c}{B: 1*3}   & 32  & 0.43  \\ 			
			\cline{1-5}
			
		\end{tabular}
		\caption{Drone Payload Packing Configuration and Schedule}
		\label{dcp}
	\end{table}
	The column "Container" means the container charging for the appointed hospital: No.1 means $(18.20^\circ N, 65.97^\circ W)$, No.2 means $(18.20^\circ N, 66.50^\circ W)$, No.3 means $(18.20^\circ N, 65.97^\circ W)$. The "*3" in column "Drone" means three day's quantity. Distance to CMC is not the shortest path since there are a mountain high more than 3000 m. From column "Flight Time" we can see the quickest time for finishing the delivery task is 0.55 hour. \textcolor[rgb]{1.00,0.00,0.00}{Now, Problem \ref{P3} has been solved.}
	   
	Now the numbers and types of basic drones and medical kits for three delivery tasks are clear, we can focus on how to pack the cargo into three containers so that more area can be reconnoitred under limited resource. For achieving the max spatial utilization rate, we regard this problem as a three-dimensional packing problem and refer to the following algorithm \cite{6} to get the answer of containers configuration. Two related algorithms are shown in appendix A. Finally, we obtain the final configuration of ISO cargo containers as shown in Table \ref{pctc}.  \textcolor[rgb]{1.00,0.00,0.00}{Now, Problem \ref{P1} has been solved.}
%	\begin{algorithm}[htb]
%		\setstretch{1.35} %??????????????????1.35??
%		\caption{}
%		\label{alg:Framwork}
%		\begin{algorithmic}
%			\REQUIRE ~~\\
%			The set of positive samples for current batch, $P_n$;\\
%			The set of unlabelled samples for current batch, $U_n$;\\
%			Ensemble of classifiers on former batches, $E_{n-1}$;
%			
	
%			\ENSURE ~~\\
%			
%		\end{algorithmic}
%	\end{algorithm}




\begin{algorithm}
	\label{al1}
	\renewcommand{\algorithmicrequire}{\textbf{Input:}}
	\renewcommand{\algorithmicensure}{\textbf{Output: }}
	\caption{3D Bin Packing Heuristic Algorithm}
	\begin{algorithmic}[1]
		\REQUIRE  Size of containers, drones and medical packages; Number of indispensable containers and medical packages for three day's delivery task.
		\ENSURE Loaded cargo quantity with the types.
		\STATE Denote the set of n items as I. Each item is of length $l_i$, height $h_i$ and width $w_i$. 
		\STATE Initialize a sufficiently large bin(B) with length L, width W, height H (We could set 
		$L = W = H = \sum\limits_{i = 1}^n {\max ({l_i},{h_i},{w_i})} $).
		\STATE Initialize the set of remaining items $I' = I$ and the set of empty maximal spaces as $ES = \phi$.
		\FORALL {$t$ = 1 to $n$}
		\STATE   \textbf{if} t=1 \textbf{then}
		\STATE   Select an item with largest surface area.
		\STATE Put the item into the    bin and generate 3 empty maximal spaces ES1. Update ES = ES1.
		\STATE  \textbf{else}
		\STATE Select a item i from set S according to least waste space heuristic(Algorithm 3). Update 
		$ I' \leftarrow I'\backslash i$.
		\STATE  Select an empty maximal space from ES and decide the orientation according to Least surface Area Heuristic(Algorithm 2). 
 		\STATE  Generate new empty maximal spaces(ES1) and delete those that are intersected and overlapped(ES2). Update $ES \leftarrow ES \cup ES1\backslash ES2$.
 		\STATE \textbf{end if}
		\ENDFOR
		\STATE \textbf{Return} loaded cargo quantity with the types.

	\end{algorithmic}  
\end{algorithm}

\begin{table}[H]
	\centering

	\begin{tabular}{llll}
		\cline{1-4}
		
		\multicolumn{1}{c}{Container}  & \multicolumn{1}{c}{Position} & \multicolumn{1}{c}{Drone Configuration} & \multicolumn{1}{c}{Medical Kits Configuration} \\ 
		\cline{1-4}
		
		\multicolumn{1}{c}{1} & \multicolumn{1}{c}{$(18.29^\circ N, 66.97^\circ W)$} & \multicolumn{1}{c}{B:75 G:1 H:1}    &  $M_1\times3+M_n\times n$     \\
		\multicolumn{1}{c}{2} & \multicolumn{1}{c}{$(18.20^\circ N, 66.50^\circ W)$} & \multicolumn{1}{c}{B:75 G:1 H:1}   &   $M_n \times n$      \\
		\multicolumn{1}{c}{3} & \multicolumn{1}{c}{$(18.20^\circ N, 65.97^\circ W)$}  &\multicolumn{1}{c}{B:61 C:3 E:3 H:1}   &  $M_1\times 18+M_2\times 6 +M_3 \times 12 +M_n\times n$ \\ 
		
		\cline{1-4}
		
	\end{tabular}
	\caption{Packing Configuration for Three Containers}
		\label{pctc}
\end{table}
We find there are space for containing hundreds of medical packages, so we don't give specific data about it. $M_n \times n$ means there are medical packages for more than one week's demands.
	
    
    
    \section{Optimum Localization Model (OLM)} \label{ols}
    
    \subsection{Optimum Container Number}
    The maximum flight distance among all types of drones is about 52.66 km, while Puerto Rico is about 170 km long and 60 km wide. If drones come back to their starting point, they are up to 26 km away from their starting point and then have to come back. What's more, drones' main task is to finish the reconnaissance work of several allocated grids, which needs more traversal time. By drawing a reachable circle area with a corresponding size on the map(the next few pictures show it), it's easy to discover there will be vast areas that cannot be reconnoitred if there are only one or two cargo containers. Therefore, we decide to arrange \textbf{three containers}, which may ensure 70\% of Puerto Rico can be probably reconnoitred by rough estimates. 
    
    \subsection{Optimum Locations}
    Based on equation \ref{8} $\sim$ \ref{dc}, capacity of drones and requirement of emergency medical packages, we can calculate the maximum journey distance of seven types of drones and draw a set of concentric circles centered on five hospitals with the radius are the calculated flight distances. In this way, the optimum locations must be in the coverage area since drones must be able to fly to hospitals. For making it a good appearance, Figure \ref{hcc} shows the concentric circles of three types of drones (B, C, F) since their circle areas are relatively large.  
      \begin{figure}[H]                                         
    	\centering
    	\includegraphics[width = 1\textwidth]{hcc.png}        
    	\caption{Concentric Circles Centered on Five Hospitals}                           
    	\label{hcc}                                          
    \end{figure}

	In Figure \ref{hcc}, blue circles belong to B drone, red circles belong to C drone and green circles belong to F drone. The circles for B drone centered in Hospital HIMA and for F drone centered in Puerto Rico Children's Hospital are lost because the drone can't fly too far or can't carry all the medical supply needed by the hospital in that situation.
	\par
	There is another dimension: on the basis of section \ref{ldis}, B Drone has the maximum performance. As a result, we will take B Drone into next localization calculation - cruising task is primarily responsible by B Drone. Now let's focus on the \textbf{flight paths}.
	\par
	\begin{description}
		\item[Path 1] Drones leave the container and return back after carrying out the reconnaissance mission. So the reconnaissance radius of each container is half of the maximum range of B drone. We can draw three circles with containers as the center and half of B drone's maximum range as the radius based on Figure \ref{thd2} (cover Puerto Rico and the high-weight grids as much as possible). Figure \ref{circle3} is our primary expectation.
		Here we have the inequality:
		\begin{equation}
		s=v*t_{max} \ge 2*s1 + s2   
		\end{equation}  
		Please note that here the drone is unloaded so the maximum journey distance s is represented by $v*t_{max}$ straight. 
		
		  \begin{figure}[H]                                         
			\centering
			\includegraphics[width = 1\textwidth]{tcircle.png}        
			\caption{An Expected Localization Based on B Done}                           
			\label{circle3}                                          
		\end{figure}
		 \item[Path 2] Drones set out from a container and land in the contiguous container. In the figure above, if the two circles intersect or are tangent, some drones may have such flight trajectories, but this does not expand our scope of reconnaissance. Here we have the inequality:
		 		\begin{equation}
		 s=v*t_{max} \ge s1 + s2 + s3  
		 \end{equation}  
		 
		 \item[Path 3] Drones start out from a container and land in the hospital after finishing the reconnaissance missions (Assumption 6).  
		 \begin{Theorem} \label{thm:latex}
		 	The distance from any point P on the ellipse to the two focal points F1, F2, is equal to twice the length of the long axis $a$, namely $|PF_1|+|PF_2|=2a$.
		 \end{Theorem}
	 
	     According to the theorem above, for any point inside the ellipse, if we focus on the location of containers and hospitals, and take the maximum range of B drone as the long axis length - draw ellipses - then the region of all ellipses is the scope that drones can detect and eventually land in a hospital. Figure \ref{ellipses} shows the effect of one red ellipse for C drone, four green ellipses for F Drone and five blue ellipses for B Drone (use the same containers position as Figure \ref{circle3}). Figure \ref{circleE} shows the combination Figure \ref{circle3} and \ref{ellipses} which is our expected location map with the help of the grids weights concentration map (Figure \ref{thd2}). All the areas covered are reachable. 
	     
	        \begin{figure}[H]                                         
	     	\centering
	     	\includegraphics[width = 1\textwidth]{ellipses.png}        
	     	\caption{Useful Ellipses focused on Containers and Hospitals}                           
	     	\label{ellipses}                                          
	     \end{figure}
	     
	       \begin{figure}[H]                                         
	     	\centering
	     	\includegraphics[width = 1\textwidth]{circleE.png}        
	     	\caption{Combination of Figure \ref{circle3} and \ref{ellipses} }                          
	     	\label{circleE}                                          
	     \end{figure}
	\end{description}

	In order to find the best location of three containers and evaluate our locations result, we set up two evaluation indicators: coverage ratio of grids number ($\alpha$), coverage ratio of grids weights ($\beta$). Their formula for calculation is as follows.
	\begin{align}
	\alpha&= \frac{C}{N} \\
	\beta&=\frac{{\sum\limits_{i = 1}^N {{G_i}{W_i}} }}{{\sum\limits_{i = 1}^N {{W_i}} }}
	\end{align} \par
   $G_i$ means the grid $i$, and $W_i$ means the weight or score of grid $i$. By traversing some probable locations and calculating $\alpha$ and $\beta$, we eventually locate three containers as the optimum localization: ( $(18.29^\circ, -66.97^\circ), (18.20^\circ, -66.50^\circ), (18.20^\circ, -65.97^\circ)$ ). There is about 0.005 degree deviation. Figure \ref{finallocation} shows the visual image. In this solution, the $\alpha=82.49\%$, the $\beta = 86.08\%$. Some blank areas are not covered because the weights of the grids are relatively small. More specifically, there maybe low population density, low traffic density or abominable geographical environment. \textcolor[rgb]{1.00,0.00,0.00}{Now, Problem \ref{P2} has been solved.}   
   
    \begin{figure}[H]                                         
   	\centering
   	\includegraphics[width = 0.9\textwidth]{fcircleE.png}        
   	\caption{Final Optimum Localization}                          
   	\label{finallocation}                                          
   \end{figure}
     
      \section{Solution for Drone Flight Plan Based on Above Solutions  }

We stimulate the flight plan using \textbf{A*} Algorithm. Use the grids weights matrix as the Planning Environment Model of \textbf{A*}, and a grid can be a searching node of \textbf{A*}, since setting reasonable heuristic functions can benefit the selection of the node with minimum generation value as the node to be searched next. The essence of A* algorithm is the combination of greedy algorithm and heuristic search algorithm, so A* algorithm combines the advantages of both, and meanwhile inherits the characteristics of both. As a heuristic algorithm, A* algorithm has high search efficiency in path planning.  \par 
For A* algorithm, reasonable cost function design is the most important work. In the problem of route planning in drone defection, we consider that there are three quantities which have important influence on the final trace, then we design the cost function as follows: 
\begin{equation}
f(n)=g(n)+h(n)-m(n)
\end{equation}

 f (n) denotes the cost of the present node, g (n) denotes the distance traveled from the starting point to the present node, and m (n) denotes the total score obtained from the starting point to the present node. \par
 
  \begin{figure}[H]                                         
 	\centering
 	\includegraphics[width = 0.8\textwidth]{weightGrade.png}        
 	\caption{Grade Division Based Grids Weights Matrix (Figure \ref{finalweights})}                          
 	\label{gdbo}                                          
 \end{figure}
We normalize the weights in the grid weight matrix above and get a fractional matrix W in the fractional interval mapped to [0,6] (Figure \ref{gdbo}). Drones needs to move 5 km to fly over a grid horizontally or vertically, and 7 km to fly obliquely over a grid. Whatever way it flies over a grid, the score of the grid is equal to its original score minus three (the score is at least 0, it will not become negative). On this basis, we give: 
\begin{align}
g(j) &= g(i) + \sqrt {{{({x_i} - {x_j})}^2} + {{({y_i} - {y_j})}^2}} \\
h(j) &= \sqrt {{{({x_E} - {x_j})}^2} + {{({y_E} - {y_j})}^2}} \\
m(j) &= m(i) + \left\{ \begin{array}{l}
3,if{w_j} \ge 3\\
{w_j},if{w_j} < 3 
\end{array} \right\}
\end{align}
 I is the current node and j is a neighbor node of I. E is the midpoint and W is the fraction. \par
 The implementation process and summary of A* algorithm in path planning are as follows: starting from the starting point, searching the point with the lowest cost function value in the neighborhood node, expanding gradually outward until the end of the algorithm search. Firstly, starting from the starting point and the starting point is stored in the open set and all the nodes in its neighborhood are scanned to calculate the $f(j)$ value of their grid nodes in all the neighborhood nodes and coexist. The $f(j)$ calculated are stored in the open set. After completing the node expansion of the starting point, we delete the starting point from the open set and join it into the closed set. Select the candidate node with the lowest $f(j)$ value in the open set as the temporary path point, and point the pointer of the current node to the new node, then we complete the establishment of the second temporary path point. After that, we start the new node expansion. The next is the process for realization \cite{7}.
 
     	    \begin{algorithm}
 	\label{al4}
 	\renewcommand{\algorithmicrequire}{\textbf{Input:}}
 	\renewcommand{\algorithmicensure}{\textbf{Output: }}
 	\caption{Pseudo Code for Flight Plan}
 	\begin{algorithmic}[2]
 		\REQUIRE  The weight matrix $W$.
 		\ENSURE The path set of drone start from three containers. 
 		\STATE  Step1: Denote the P container as the starting point. 
 		\STATE Step2: Select a point within the container coverage as the end point of the A* algorithm. 
 		\STATE Step3: Run A* algorithm to get the best path and record the score mE of the path 
 		\STATE Step4: Repeat Step3 and Step4 to find the path with the highest score, record the path, and modify the score of the node through which the path passes. 
 		\STATE Step5: Repeat Step5 until you find all the paths starting from this container. 
 		\STATE Step6: Repeat Step2 to get the path set of three containers. 
 		\STATE \textbf{return} the Path Set.
 	\end{algorithmic}  
 \end{algorithm}
The result of running this algorithm is: $\alpha$=71.27\%, $\beta$= 83.61\%.

Taking terrain factors into account, we mark some unreachable areas with yellow and set their grades to zero. Then we get Figure \ref{gcla}.
 \begin{figure}[H]                                         
 	\centering
 	\includegraphics[width = 0.9\textwidth]{detailGrids.png}        
 	\caption{Grids Classification}                          
 	\label{gcla}                                          
 \end{figure}
We give priority to high-ranking grids and do resource allocation mainly on grids with grade 5 and 6, only to find no more than 20 B drones are needed for each containers. Figure \ref{pfp} shows the detail. 
\begin{figure}[H]                                         
	\centering
	\includegraphics[width = 1\textwidth]{plan.jpg}        
	\caption{A Part of Fight Plan Covering Grids of Grade 5 and 6}                          
	\label{pfp}                                          
\end{figure}
    
    The next figure are concrete flight routes starting from three containers. The thicker the line is, the drone fleet bigger. \textcolor[rgb]{1.00,0.00,0.00}{Now, Problem \ref{P4} has been solved.}   
    \begin{figure}[H]                                         
    	\centering
    	\includegraphics[width = 1\textwidth]{routes.png}        
    	\caption{A Sketch Map Showing Drone Routes}                          
    	\label{pfpss}                                          
    \end{figure}
    
%            \begin{table}[H]
%        	\centering
%        	\caption{The Different Layer of The Forty Countries}
%        	\label{ceng}
%        	\begin{tabular}{cccccccll}
%        		\toprule
%        		Hierarchy                & \multicolumn{5}{c}{Country}                                   \\
%        		\midrule
%        		The First Network Layer  & America      & Mexico      & Russian     & Ukraine  & India     \\
%        		& Germany      & China       &             &          &          \\
%        		\midrule
%        		The Second Network Layer & Bangladesh   & Pakistan    & U.K         & France   & Canada    \\
%        		& Italy        & Philippines & Iran        &          &           \\
%        		\midrule
%        		The Third Network Layer  & Turkey       & Spain       & Afghanistan & Algeria  & Poland   \\
%        		& Morocco      & Japan       & Viet Nam    & Korea    &           \\
%        		\midrule
%        		The Forth Network Layer  & Brazil       & Colombia    & Argentina   & Iraq     & Congo   \\
%        		& South Africa & Nigeria     & Thailand    & Tanzania & Myanmar   \\
%        		& Indonesia    & Sudan       & Egypt       & Kenya    & Uganda   \\
%        		& Ethiopia     &             &             &          &          \\
%        		\bottomrule
%        	\end{tabular}
%        \end{table}
%    
%        \begin{table}[H]
%        	\centering
%        	\caption{The Predictions of 6th-16th Language`s Total Numbers of Speakers}
%        	\label{Bdaan}
%        	\begin{tabular}{ccccccc}
%        		\toprule
%        		& \multicolumn{2}{c}{L1} & \multicolumn{2}{c}{L2} & \multicolumn{2}{c}{Total} \\
%        		\midrule
%        		& Now    & In 50 Years   & Now    & In 50 Years   & Now     & In 50 Years     \\
%        		Malay      & 77     & 107           & 204    & 271           & 281     & 378             \\
%        		Bengali    & 242    & 340           & 19     & 22            & 261     & 362             \\
%        		Russian    & 153    & 183           & 113    & 158           & 267     & 341             \\
%        		Portuguese & 218    & 297           & 11     & 16            & 229     & 313             \\
%        		French     & 76     & 101           & 153    & 203           & 229     & 304             \\
%        		Hausa      & 85     & 132           & 65     & 93            & 150     & 225             \\
%        		Punjabi    & 148    & 192           &        &               & 148     & 192             \\
%        		German     & 76     & 106           & 52     & 69            & 129     & 175             \\
%        		Persian    & 60     & 84            & 61     & 85            & 121     & 169             \\
%        		Japanese   & 128    & 164           & 1      & 1.3           & 129     & 165.3           \\
%        		Swahili    & 16     & 22            & 91     & 131           & 107     & 153\\
%        		\bottomrule
%        	\end{tabular}
%        \end{table}
    
    
    \section{Model Rationality  Analysis}
    
    \begin{itemize}
    	\item Grids number coverage rate $\alpha$ is smaller than grids weight coverage rate, which validate our grids weights matrix in \textbf{GTM} (section \ref{finalweights}).
    	\item Each grid is a square 5 km on a side while the breadth of vision of drones is about 2770 m. A drone only needs to have a 10 km flight for reconnoitring a whole grid.
    	\item The results indicate our whole packing configuration can achieve the entire reconnaissance task of the marked area in Figure \ref{finallocation}, which means our models and solutions are reliable.
    	\item \textbf{Sensitivity Analysis}: We reconsider the factors affecting the grids - calculate grids weights with only one factor independently - and finally obtain the following comparison map.
    	    \begin{figure}[H]                                         
    		\centering
    		\includegraphics[width = 0.8\textwidth]{exam.png}        
    		\caption{Sensitivity Analysis on Different Affecting Factor of Grids Weights}                          
    		\label{pf}                                          
    	\end{figure}
    \end{itemize}
    
%    \subsection{Error Analysis}
%     \begin{itemize}
%    	\item
%    	\item
%    	\item
%    \end{itemize}
    
    
    \section{Strengths and Weaknesses}
    \subsection{Strengths}
    \begin{itemize}
    	\item The grids traversing model (GTM) is a basis of all the models, which achieves the quantization of real problems properly. 
    	\item GTM adopts traffic highway density, population density and topographic factor to evaluate the weights of grids reasonably.
    	\item The entropy weighting model (EWM) makes full use of information entropy to weigh different indexes scientifically, which is propitious to identify pivotal grids and screen out meaningless area. 
    	\item The optimum localization model (OLM) centers on hospital and draw ellipses and circles, considering most of the probabilities and converting their into math model.
    \end{itemize}
    \subsection{Weaknesses}
    \begin{itemize}
    	\item Enormous uncertain variables entail our models hard to be examined.
    	\item It's difficult to realize GTM and solve the flight plan perfectly under such limited time. 
    	\item The final solution of packing configuration, and localization is given too roughly and lacking validation, while the detailed process is too tedious to say. 
    \end{itemize}

\begin{thebibliography}{99}                
	\bibitem{1} Zhenji Ding. Research on Emergency Relief Task Assignment Technology of Multi-UAV in Urban Environment [D].Nanjing: Nanjing University of Aeronautics and Astronautics, 2016. 
	\bibitem{2} Yaqian Zhang. Study on Road Network Density Based on Land Volume Ratio[D]. JiNan: Shandong Jianzhu University, 2018.
	\bibitem{3} \href{https://en.wikipedia.org/wiki/Municipalities_of_Puerto_Rico#cite_note-census-2010-11?tdsourcetag=s_pcqq_aiomsg} {Population Data of Puerto Rico's City} (This is a hyperlink and you can click it directly.)
	\bibitem{4} \href{https://en.wikipedia.org/wiki/Puerto_Rico#Population_makeup}{Puerto Rico's Total Population Data Over the Years}
	\bibitem{5} Guanhong Wu, Jiaming Zhu. Evaluation of Urban Livability in Huaihai Economic Zone Based on Entropy Weight Method [J].Journal of Hunan City College, 2018, 27(3): 42-45. 
	\bibitem{6} Haoyuan Hu, Xiaodong Zhang. Solving a New 3D Bin Packing Problem with Deep Reinforcement Learning Method. Hangzhou, 2017.
	\bibitem{7} Zhonghua Han, Xinghao Feng, Lu Zhe, Yang liying, An Improved UAV Path Planning Environment Modeling Method[j], Information and Control,2018,47(03):371-378.
\end{thebibliography}

\section{Memo}

\noindent TO: Dear Madam  \\
From: A Modeling Team  \\
Subject: Results, Conclusions and Recommendations about 'DroneGo'  \\

Dear madam, we are honored to contribute to the disaster relief work you have done. Here are our results, conclusions and recommendations about 'DroneGo' system designed by our model.

\textbf{Results}
To develop a DroneGo disaster response system to support the Puerto Rico hurricane disaster scenario, we need three ISO cargo con trainers to cover most of the islands (area coverage: 82.49\% and weighted area coverage: 86.08\%). And one at(18.29$\pm$ 0.005$^\circ$N; 66.97$\pm$0.005$^\circ$W), one at(18.20$\pm$0.005$^\circ$N; 66.50$\pm$0.005$^\circ$W), and the last at (18.20$\pm$0.005$^\circ$N;\\ 65.97$\pm$0.005$^\circ$W).
As for medical packages allocation, we first ensure that we can supply the three-day supply of the five hospitals on the island and assign tasks to the corresponding containers according to the positioning of the containers. Based on the drone evaluation, we have determined the most suitable type of drone for packages delivery and reconnaissance. Combined with the above modeling results, the container task and the drone transport fleet are configured as follows:
	\begin{table}[H]
	\centering
	
	\begin{tabular}{lllcc}
		\cline{1-5}
		
		\multicolumn{1}{c}{Hospital (abbr.)}  & \multicolumn{1}{c}{Container}         & \multicolumn{1}{c}{Drone} & \multicolumn{1}{c}{Distance (km)}     & \multicolumn{1}{c}{Fight Time (hour)} \\
		\cline{1-5}
		
		\multicolumn{1}{c}{CMC}  & \multicolumn{1}{c}{3}  &\multicolumn{1}{c}{B: 1$\times$3}    & 36 & 0.55   \\
		\multicolumn{1}{c}{HH}   & \multicolumn{1}{c}{3}  &\multicolumn{1}{c}{E: 1$\times$3}    & 7 &  0.14  \\
		\multicolumn{1}{c}{HPS}  & \multicolumn{1}{c}{3}  &\multicolumn{1}{c}{B: 1$\times$3}  & 27 & 0.39 \\ 
		\multicolumn{1}{c}{PRCH} & \multicolumn{1}{c}{3}  &\multicolumn{1}{c}{B: 1$\times$3;  C: 1$\times$3}  & 28 & B: 0.448; C: 0.494     \\
		\multicolumn{1}{c}{HPA}  & \multicolumn{1}{c}{1}  &\multicolumn{1}{c}{B: 1*3}   & 32  & 0.43  \\ 			
		\cline{1-5}
		
	\end{tabular}
	\caption{Drone Payload Packing Configuration and Schedule}

\end{table}

When resolving reconnaissance missions, we use the B-type drone with the highest reconnaissance evaluation score to perform the reconnaissance mission according to the modeling results. Then we found that the number of unmanned aerial vehicles that can be loaded in the remaining space of the container far exceeds the model planning needs, resulting in wasted container space. See the table below for details: 

\begin{table}[H]
	\centering
	
	\begin{tabular}{llll}
		\cline{1-4}
		
		\multicolumn{1}{c}{Container}  & \multicolumn{1}{c}{Position} & \multicolumn{1}{c}{Drone Configuration} & \multicolumn{1}{c}{Medical Kits Configuration} \\ 
		\cline{1-4}
		
		\multicolumn{1}{c}{1} & \multicolumn{1}{c}{$(18.29^\circ N, 66.97^\circ W)$} & \multicolumn{1}{c}{B:75 G:1 H:1}    &  $M_1\times3+M_n\times n$     \\
		\multicolumn{1}{c}{2} & \multicolumn{1}{c}{$(18.20^\circ N, 66.50^\circ W)$} & \multicolumn{1}{c}{B:75 G:1 H:1}   &   $M_n \times n$      \\
		\multicolumn{1}{c}{3} & \multicolumn{1}{c}{$(18.20^\circ N, 65.97^\circ W)$}  &\multicolumn{1}{c}{B:61 C:3 E:3 H:1}   &  $M_1\times 18+M_2\times 6 +M_3 \times 12 +M_n\times n$ \\ 
		
		\cline{1-4}
		
	\end{tabular}
	\caption{Packing Configuration for Three Containers}

\end{table}

\textbf{Conclusions}
The system designed by our model can finish the medical supply delivery task in 0.55 hours (33 mins), which will ensure that people in the disaster area can get help in the first time when your organization arrives.
In the video reconnaissance task, it is sad that this system could not cover the whole main island of Puerto Rico. In our model, we not only consider the factors of the roads, but also the factors of the population, which make our designed system has a better performance than only considering factor of roads.

\textbf{Recommendations } \\
In order to cover more area, some schemes are proposed: 

1.	Prepare more containers. In the Puerto Rico hurricane scenario, we believe that more than 95\% area could be covered with one more container.
 
2.	Prepare better drone. According to the information we have consulted, there are drones that could fly longer distance than the drones provided in this problem. With a larger max voyage, every container could handle a large area.

3.	Set more drone landing points. The problem that there are no more options for landing points has always perplexed us. Drones will not have to fly back to the containers any more with more landing points, which will double the voyage of the drones.

4.	The drone fleet and medical packages contained in each container has exceeded demand. Smaller containers will be more suitable.
 
5.	If this system is going to be used in other scenarios, local topography is an important factor to be considered to decide the locations.

\newpage
    \begin{appendices}
    	\section{Algorithms 3 and 4}
%    	\subsection{Least Surface Area Heuristic}
    	    \begin{algorithm}
    		\label{al2}
    		\renewcommand{\algorithmicrequire}{\textbf{Input:}}
    		\renewcommand{\algorithmicensure}{\textbf{Output: }}
    		\caption{Least Surface Area Heuristic}
    		\begin{algorithmic}[2]
    			\REQUIRE  The set of collection of residues ($I$), volume of cargo ($V$).
    			\ENSURE The item i to be loaded.
    			\STATE  Denote the set of empty maximal space as $ES$, the set of orientations as $O$. 
    			\STATE Initialize the least surface area for item i as $LSA_i=3*(max(l_i,w_i,h_i)*n)^2$.
    			\STATE Initialize best empty maximal space $s'_i=null$, best orientation as $o'_i = null$. 
    			\FORALL {each empty maximal space $S \in ES$ }
    			\FORALL {each orientation $o\in O$}
    			\STATE Calculate the surface area  $SA_{}i,s,o$  after putting item i in empty maximal space s with orientation o. 
    			\STATE   \textbf{if} $SA_{i,s,o} < LSA_i$ \textbf{then}
    			\STATE   Update $s'=s, o'=o$ and $LSA_i\leftarrow SA_{i,s,o}$.
    			\STATE \textbf{else if} $SA_{i,s,o}+LSA_i$ \textbf{then}
    			\STATE  Apply the tie-breaking rule. (Selecting s,o if and only if min(length(s)-o($l_i$), width(s)-o($W_i$), height(s)-o($h_i$)) is less than min(length($s_i$)-$o_i(l_i)$, width($s_i$)- $o_i(w_i)$, height($s_i$)-$o_i(h_i))$, where length(s), width(s), height(s) represents the length, width, height of empty maximal space s and o($l_i$), o($w_i$), o($h_i$) represents the length, width, height of item i with orientation o.) 
    			
    			\STATE \textbf{end if}
    			\ENDFOR
    			\ENDFOR
    			\STATE \textbf{return} $s'_i, o'_i$ for item i.
    		\end{algorithmic}  
    	\end{algorithm}
    	
%    	\subsection{Least Waste Space Heuristic}
    	\begin{algorithm}
    		\label{al3}
    		\renewcommand{\algorithmicrequire}{\textbf{Input:}}
    		\renewcommand{\algorithmicensure}{\textbf{Output: }}
    		\caption{Least Waste Space Heuristic}
    		\begin{algorithmic}[1]
    			\REQUIRE Remaining items as I', volume of item i as $V_i$.
    			\ENSURE Location and orientation of items in space.
    			\STATE  Initialize the best item $i' =null$.
    			\STATE  Initialize the least volume as $LV = {(max({l_i},{w_i},{h_i}){\rm{\cdot}}n)^3}$
    			\FORALL {each item $i\in I'$ }
    			\STATE   Calculate the volume V after packing item i according to $s'_i, o'_i$ (which is determined by Least Surface Area Heuristic). Denote the least waste space of item $LWV_i = V-V_i$. 
    			\STATE   \textbf{if} $LWV_i<LV$ \textbf{then}
    			\STATE Update $LV=LWV_i$.
    			\STATE Update $i'=i$.
    			\STATE \textbf{end if}
    			
    			\ENDFOR
    			
    			\STATE \textbf{return} $s'_i, o'_i$.
    		\end{algorithmic}  
    	\end{algorithm}
    
    	\section{Codes in Python}
    	
    	The following codes are \textbf{parts} of related programs written in Python.  \par
    \textbf{Draw Map of Puerto Rico with Grids: }
    \lstinputlisting[language=Python,breaklines]{./code/feature.py} \par
      \textbf{3D Packing Optimization Problem:} (Cite and modify from resource on Github ) 
    \lstinputlisting[language=Python,breaklines]{./code/main_copy.py}
     \textbf{Calculating Weights of Grids and Drones:} 
    \lstinputlisting[language=Python,breaklines]{./code/weight.py}
    \lstinputlisting[language=Python,breaklines]{./code/UAV_weight.py}
    \section{Codes in Matlab}
    The following codes are \textbf{parts} of related programs written in Matlab.  \\
    
%    	\textbf{\textcolor[rgb]{0.98,0.00,0.00}{Input matlab source:}}
%    \lstinputlisting[language=Matlab]{./code/mcmthesis-matlab1.m}
     \textbf{Draw pictures:} 
     \lstinputlisting[language=Matlab,breaklines]{./code/TransferAFiletoThreeColumns.m}
    \lstinputlisting[language=Matlab,breaklines]{./code/humanDensity.m}
    \lstinputlisting[language=Matlab,breaklines]{./code/TrafficDensityMap.m}
    \lstinputlisting[language=Matlab,breaklines]{./code/weight_map.m}
    
    \section{Codes in C++}
    The following codes are related programs written in C++. \par
    \textbf{Generate parameters for python drawing:} 
    \lstinputlisting[language=C++,breaklines]{./code/stringForPython.cpp}
    	
    \end{appendices}

\end{document}


\end{document}
