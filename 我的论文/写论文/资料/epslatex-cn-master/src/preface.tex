%preface.tex

\section*{译序}

本文档是Keith Reckdahl所著``Using Imported Graphics in \LaTeX\ and \pdfLaTeX'' 的中文译本。

Keith Reckdahl于1997年原著有文档``Using Imported Graphics in \LaTeXe{}''
(版本2.0,以下简称为 epslatex2),
并且已经由王磊前辈在2000年将其翻译成《\LaTeXe{} 插图指南》。
该译本由于质量上乘,历史悠久,并且作为帮助文档附在CTEX套装内发行,
因此在国内流传甚广、影响力甚大。

然而,也正是因为历史悠久,所以出现了很多与当前 \LaTeX{} 插图技术脱节的地方。
例如只讨论 \prgname{latex}+\prgname{dvips} 模式下 \file{eps} 图像格式的插入,
以及使用 \pkg{caption2} 等过时宏包进行插图标题的处理等等。
为此,原作者在2006年将epslatex2更新为版本3.0.1(以下简称为 epslatex3)。
虽然关于 \LaTeX{} 插图这一话题有不少书籍和电子文档都有相关介绍,
不过即便距今已十年有余,但英文版epslatex3仍可以说是最详尽的参考文档。
而国内除了一些出版的书籍外,尚没有一份全面的专题文档介绍关于插图的各方面知识。
这也正是翻译epslatex3的动机。

随着这些年 \LaTeX{} 的发展,特别是相关宏包以及 \XeTeX{}、\LuaTeX 等新编译引擎的出现和发展,
2006年版本的epslatex3也有不少已经过时的知识。
因此在翻译过程中也加入了译者自己的一些修改。
较大的改动有:
\begin{itemize}
	\item 原文档第7节只讨论了 \prgname{dvips} 和 \prgname{pdftex} 两种引擎程序。
	这里加入了 \prgname{dvipdfmx}、\prgname{xetex}和 \prgname{luatex}等更多编译方式。
	\item 原文档12.1节提到了 \pkg{overpic} 宏包,
	本文档基于旧译本,在第~\ref{ssec:overpic} 节加以详细介绍。
    \item 在第~\ref{sssec:esopic} 节简要介绍了 \pkg{eso-pic} 宏包用法。
	\item 原文档的第32节使用 \pkg{subfig} 宏包处理子图标题,
	这里代之以介绍 \pkg{subcaption} 宏包。
	\item 第~\ref{sec:figintext} 节关于图文混排的内容来自于旧译本。
\end{itemize}
对于明显过时的章节将在行文处加以说明,但出于完整性考虑仍然翻译出来供参考。
此外,行文各处的细节修改则通过脚注的方式给出。

本文档主要是在王磊前辈的译文基础上,
结合英文版epslatex3对epslatex2的变动翻译而来。
考虑到目前的 \LaTeX{} 编译程序并非只有 \pdfTeX{},
因此参照旧译本仍命名为《\LaTeXe{} 插图指南》。
继续使用或修改了旧译本中添加的内容已征得原译者许可。
在翻译过程中还参考了刘海洋的著作《\LaTeX{} 入门》以及其它相关宏包的文档,
在此一并表示感谢。

本文档可在 \LaTeX\ 项目公共协议(LPPL, \LaTeX\ Project Public License)下复制和分发。
LPPL 协议的内容见 \url{http://www.latex-project.org/lppl/}。

\begin{flushright}
盛文博\\
\href{mailto:wbsheng88@foxmail.com}{<wbsheng88@foxmail.com>}\\
二零一七年八月
\end{flushright}

\newpage
\section*{前言}
\label{sect:preface}

在用 \LaTeX\ 编写论文、书稿时,经常要遇到使用图形的情况。
本书将讲述如何在 \LaTeX\ 文件中插入图像以及其它一些相关问题。\footnote{
    原文档涉及到图片的单词包括“figure”和“graphic”,
    前者指的是 \LaTeX{} 文档中作为独立单元的图元素,而后者更偏向于来自于外部的图片文件及图片内容。
    本译文中将 figure 和 graphic 分别翻译为“图形”和“图像”。——译注}
由于完全阅读本书需要较多的时间,
如果想要查询某一特定的信息,
那么可以通过\hyperref[toc]{目录}和\hyperref[sec:index]{索引}直接阅读相关内容。

插图的基本步骤是,首先确认宏包的导入
\begin{lstlisting}
\usepackage{graphicx}
\end{lstlisting}
然后使用 \cmd{includegraphics} 命令来插图
\begin{lstlisting}
\includegraphics{file}
\end{lstlisting}
该命令的更多细节将在第~\ref{sec:graphicsinclusion} 节介绍。

本文档分为如下的五个部分。

\paragraph{第一部分:背景介绍}
这一部分介绍了一些有关的历史资料和基本的 \LaTeX{} 术语。
此外还包括
\begin{itemize}
	\item Encapsulated PostScript (\file{eps}) 图形格式,\file{eps} 与 \file{ps} 文件的区别,
	以及将其它图形格式转为 \file{eps} 格式的方法。
	\item 通过 \pdfTeX 引擎可以直接导入的图片格式(\file{jpeg}, \file{png}, \file{pdf}, \MetaPost)。
	\item 一些与图形有关的自由软件和共享软件。
\end{itemize}

\paragraph{第二部分:\LaTeX{} 图形宏集}
这一部分详细介绍了 \LaTeXe{} 图形宏集中用于导入、缩放和旋转图形的命令。
这部分涵盖了 \LaTeXe{} 图形宏集文档的大部分内容\cite{grfguide}。

\paragraph{第三部分:\LaTeXe{} 图形命令的使用}
这一部分介绍了如何使用 \LaTeXe{} 图形宏包套件中的命令来引入、缩放和旋转图形。
此外还讨论了以下三种情况:
\begin{itemize}
	\item 在支持管道(例如 UNIX )的系统中,使用 \prgname{dvips} 还可以实时地插入压缩的 EPS 图形或其它格式的图形(\file{tiff}, \file{gif}, \file{jpeg}, \file{pict} 等)。
	在其它的系统中,非 \file{eps} 格式图形必须先转换为 \file{eps} 才行。
	因为无论 \LaTeX\ 还是 \prgname{dvips} 都没有内置解压缩和转换图像格式转换的功能,
	所以相应软件需要使用者提供。
	\item 由于许多图形应用程序只支持 \ascii 文本格式,
	\pkg{psfrag} 宏包可以将 \file{eps} 图形中的文字替换为 \LaTeX{} 符号或数学表达式。
	\item 当一个\file{eps}图形被多次使用时(比如文字后面或页眉上的徽标),最后生成的 \file{ps} 文件会包含此\file{eps}图形的很多个副本。
	当所使用的图形不是位图格式时,可以通过定义 PostScript 命令来避免此图形被重复插入,
	从而减小得到的\file{eps}文件体积。
\end{itemize}

\paragraph{第四部分:\env{figure} 环境}
将插入的图形放置于\env{figure} 环境中有几个优点:
在\env{figure}环境中的图形会被自动编号,从而可以引用或添加到目录中;
置于该环境中的图形可以通过浮动实现较好的分页,所以可以很容易制作出具有专业水准的文稿。
除了介绍\env{figure}环境的一般信息外,这一部分还讲述了如下一些和图形有关的话题:
\begin{itemize}
	\item 怎样自定义\env{figure}环境,例如怎样调整图形的位置、图形周围的距离、
	标题的距离,以及在图形与文本之间加入分隔线等。
	还可以自定义标题的格式,改变标题的样式、宽度和字体等。
	\item 怎样创建边注图形和伸入边注位置的宽图形。
	\item 怎样在纵向页面版式的文档中加入横向图形。
	\item 怎样将标题放置于图形的两边而不是上下。
	\item 对于双面排版的文档,怎样确保一幅图形放置于奇数页或偶数页。
	还有,怎样确保两幅图形都出现在同一对开页面上。
	\item 怎样得到带框的图形。
\end{itemize}

\paragraph{第五部分:复杂图形}
这一部分介绍如何创建较为复杂的图形,用以同时包含多个图像。
\begin{itemize}
	\item 怎样形成并列的图形、图像和子图。
	\item 怎样在一个浮动体内将一个表格放置于图形之后。
	\item 怎样堆叠多行图形。
	\item 怎样创建跨页的连续图形。
\end{itemize}

\subsection*{如何得到本文档}
本书的 \file{pdf}格式和 \file{ps}格式的英文原版网络位置为\footnote{
此段有删改。——译注}
\begin{center}
	\href{http://ctan.tug.org/tex-archive/info/epslatex/english/epslatex.pdf}{CTAN/info/epslatex/english/epslatex.pdf}\\
    \href{http://ctan.tug.org/tex-archive/info/epslatex/english/epslatex.ps}{CTAN/info/epslatex/english/epslatex.ps}
\end{center}
本文档版本2.0的法文版本由Jean-Pierre Drucbert完成,
其 \file{pdf} 和 \file{ps} 文件可以从如下位置获得
\begin{center}
    \href{http://ctan.tug.org/tex-archive/info/epslatex/french/fepslatex.pdf}{CTAN/info/epslatex/french/fepslatex.pdf} \\
    \href{http://ctan.tug.org/tex-archive/info/epslatex/french/fepslatex.ps}{CTAN/info/epslatex/french/fepslatex.ps}
\end{center}
完整的 CTAN (Comprehensive \TeX{} Archive Network) 映像站点可以从 \url{https://ctan.org/mirrors} 获得。

\subsection*{致谢}
略。

\subsection*{许可协议}
版权所有 \copyright{} 1995--2006  Keith Reckdahl.
可在 \LaTeX\ 项目公共协议(LPPL, \LaTeX\ Project Public License)下复制和分发。

LPPL 协议的内容可参见 \url{http://www.latex-project.org/lppl/}。


\clearpage
\endinput
